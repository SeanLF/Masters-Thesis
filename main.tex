%%%%%%%%%%%%%%%%%%%%%%%%%%%%%%%%%%%%%%%%%
% Masters Thesis 
% LaTeX Template
% Version 2.5 (27/8/17)
%
% This template was downloaded from:
% http://www.LaTeXTemplates.com
%
% Version 2.x major modifications by:
% Vel (vel@latextemplates.com)
%
% This template is based on a template by:
% Steve Gunn (http://users.ecs.soton.ac.uk/srg/softwaretools/document/templates/)
% Sunil Patel (http://www.sunilpatel.co.uk/thesis-template/)
%
% Template license:
% CC BY-NC-SA 3.0 (http://creativecommons.org/licenses/by-nc-sa/3.0/)
%
%%%%%%%%%%%%%%%%%%%%%%%%%%%%%%%%%%%%%%%%%

%----------------------------------------------------------------------------------------
%	PACKAGES AND OTHER DOCUMENT CONFIGURATIONS
%----------------------------------------------------------------------------------------

\documentclass[
12pt, % The default document font size, options: 10pt, 11pt, 12pt
oneside, % Two side (alternating margins) for binding by default, uncomment to switch to one side
english, % ngerman for German
draft=false,
doublespacing, % Single line spacing, alternatives: onehalfspacing or doublespacing
%nolistspacing, % If the document is onehalfspacing or doublespacing, uncomment this to set spacing in lists to single
liststotoc, % Uncomment to add the list of figures/tables/etc to the table of contents
toctotoc, % Uncomment to add the main table of contents to the table of contents
parskip, % Uncomment to add space between paragraphs
%nohyperref, % Uncomment to not load the hyperref package
headsepline, % Uncomment to get a line under the header
%chapterinoneline, % Uncomment to place the chapter title next to the number on one line
%consistentlayout, % Uncomment to change the layout of the declaration, abstract and acknowledgements pages to match the default layout
]{MastersDoctoralThesis} % The class file specifying the document structure

\usepackage[utf8]{inputenc} % Required for inputting international characters
\usepackage[T1]{fontenc} % Output font encoding for international characters

\usepackage{mathpazo} % Use the Palatino font by default
\usepackage{graphicx}

\usepackage[backend=bibtex,natbib=true]{biblatex} % Use the bibtex backend

\addbibresource{bibliography.bib} % The filename of the bibliography

\usepackage[autostyle=true]{csquotes} % Required to generate language-dependent quotes in the bibliography

\usepackage[ruled, linesnumbered,noend]{algorithm2e} % For algorithms

% for function graphs
\usepackage{tikz}
\usepackage{color}
\usetikzlibrary{datavisualization}
\usetikzlibrary{datavisualization.formats.functions}

\usepackage{hhline}
\usepackage{rotating}
\usepackage{multirow}
\usepackage{longtable}
\usepackage{lscape}
\usepackage{adjustbox}
\usepackage{array}
\usepackage{booktabs}

\newcolumntype{R}[2]{%
    >{\adjustbox{angle=#1,lap=\width-(#2)}\bgroup}%
    l%
    <{\egroup}%
}
\newcommand*\rot{\multicolumn{1}{R{90}{1em}}}

\usepackage{dingbat}

\newcommand{\blueline}{\raisebox{2pt}{\tikz{\draw[-,black!40!blue,solid,line width = 0.9pt](0,0) -- (5mm,0);}}}
\newcommand{\yellowline}{\raisebox{2pt}{\tikz{\draw[-,black!40!yellow,solid,line width = 0.9pt](0,0) -- (5mm,0);}}}
\newcommand{\redline}{\raisebox{2pt}{\tikz{\draw[-,black!40!red,solid,line width = 0.9pt](0,0) -- (5mm,0);}}}

% for math conditions
\usepackage{amsmath}

\SetKwProg{Fn}{function}{}{}

%----------------------------------------------------------------------------------------
%	MARGIN SETTINGS
%----------------------------------------------------------------------------------------

\geometry{
	paper=letterpaper, % Change to letterpaper for US letter
	inner=2.5cm, % Inner margin
	outer=3.8cm, % Outer margin
	bindingoffset=.5cm, % Binding offset
	top=1.5cm, % Top margin
	bottom=1.5cm, % Bottom margin
	%showframe, % Uncomment to show how the type block is set on the page
}

%----------------------------------------------------------------------------------------
%	THESIS INFORMATION
%----------------------------------------------------------------------------------------

\thesistitle{Semi-Supervised Hybrid Windowing Ensembles for Learning from Evolving Streams} % Your thesis title, this is used in the title and abstract, print it elsewhere with \ttitle
\supervisor{Dr. Herna \textsc{Viktor}} % Your supervisor's name, this is used in the title page, print it elsewhere with \supname
\examiner{} % Your examiner's name, this is not currently used anywhere in the template, print it elsewhere with \examname
\degree{Master of Science in Computer Science} % Your degree name, this is used in the title page and abstract, print it elsewhere with \degreename
\author{Sean Louis Alan \textsc{Floyd}} % Your name, this is used in the title page and abstract, print it elsewhere with \authorname

\subject{Computer Science} % Your subject area, this is not currently used anywhere in the template, print it elsewhere with \subjectname
\keywords{} % Keywords for your thesis, this is not currently used anywhere in the template, print it elsewhere with \keywordnames
\university{\href{https://uottawa.ca}{University of Ottawa}} % Your university's name and URL, this is used in the title page and abstract, print it elsewhere with \univname
\department{\href{https://engineering.uottawa.ca}{School of Electrical Engineering and Computer Science\\ Faculty of Engineering}} % Your department's name and URL, this is used in the title page and abstract, print it elsewhere with \deptname
\group{\href{http://researchgroup.university.com}{Research Group Name}} % Your research group's name and URL, this is used in the title page, print it elsewhere with \groupname
\faculty{\href{http://faculty.university.com}{Faculty of Graduate and Postdoctoral Studies}} % Your faculty's name and URL, this is used in the title page and abstract, print it elsewhere with \facname

\AtBeginDocument{
\hypersetup{pdftitle=\ttitle} % Set the PDF's title to your title
\hypersetup{pdfauthor=\authorname} % Set the PDF's author to your name
\hypersetup{pdfkeywords=\keywordnames} % Set the PDF's keywords to your keywords
}

\begin{document}

\frontmatter % Use roman page numbering style (i, ii, iii, iv...) for the pre-content pages

%\pagestyle{plain} % Default to the plain heading style until the thesis style is called for the body content

%----------------------------------------------------------------------------------------
%	TITLE PAGE
%----------------------------------------------------------------------------------------

\begin{titlepage}
\begin{center}

\vspace*{.05\textheight}

\HRule \\[0.4cm] % Horizontal line
{\huge \bfseries \ttitle\par}\vspace{0.4cm} % Thesis title
\HRule \\[1.5cm] % Horizontal line
 
\large \authorname % Author name - remove the \href bracket to remove the link
 
\vfill

Thesis submitted in partial fulfilment of the requirements for the \\
\textbf{\degreename}\\[0.3cm] % University requirement text
\deptname\\\univname{}\\[2cm] % Research group name and department name
 
\vfill

{\copyright \authorname, Ottawa, Canada, 2019}\\[4cm] % Date
% \includegraphics{Logo} % University/department logo - uncomment to place it
\end{center}
\end{titlepage}

%----------------------------------------------------------------------------------------
%	ABSTRACT PAGE
%----------------------------------------------------------------------------------------
\setcounter{page}{2}
\begin{abstract}
\addchaptertocentry{\abstractname} % Add the abstract to the table of contents
In this thesis, learning refers to the intelligent computational extraction of knowledge from data. Supervised learning tasks require data to be annotated with labels, whereas for unsupervised learning, data is not labelled. Semi-supervised learning deals with data sets that are partially labelled. A major issue with supervised and semi-supervised learning of data streams is late-arriving or missing class labels. Assuming that correctly labelled data will always be available and timely is often unfeasible, and, as such, supervised methods are not directly applicable in the real world. Therefore, real-world problems usually require the use of semi-supervised or unsupervised learning techniques. In a spam detection problem, one cannot assume that it will be known, with utmost certainty, whether an email ham or spam before the user sees it. Additionally, in semi-supervised learning, "the instances having the highest [predictive] confidence are not necessarily the most useful ones" \cite{homayoun2016review}. We investigate how self-training performs without its selective heuristic in a streaming setting.

This leads us to our contributions. We extend an existing concept drift detector to operate without any labelled data, by using a sliding window of our ensemble's prediction confidence, instead of a boolean indicating whether the ensemble's predictions are correct. We also extend selective self-training, a semi-supervised learning method, by using all predictions, and not only those with high predictive confidence. Finally, we introduce a novel windowing type for ensembles, as sliding windows are very time consuming and regular tumbling windows are not a suitable replacement. Our windowing technique can be considered a hybrid of the two: we train each sub-classifier in the ensemble with tumbling windows, but delay training in such a way that only one sub-classifier can update its model per iteration.

We found, through statistical significance tests, that our framework is (roughly 160 times) faster than current state of the art techniques, and achieves comparable predictive accuracy.
That being said, more research is needed to further reduce the quantity of labelled data used for training, while also increasing its predictive accuracy.\end{abstract}

%----------------------------------------------------------------------------------------
%	ACKNOWLEDGEMENTS
%----------------------------------------------------------------------------------------

\begin{acknowledgements}
\addchaptertocentry{\acknowledgementname} % Add the acknowledgements to the table of contents
The research conducted for this thesis has been financed by the Province of Ontario and the University of Ottawa.

I would particularly like to thank my supervisor, Dr. Herna L. Viktor, for her guidance, incredible patience, and, her support.

Finally, I could not have completed this thesis without the support, and encouragement of my friends and family.

Many thanks to each and every one of you.
\end{acknowledgements}

%----------------------------------------------------------------------------------------
%	LIST OF CONTENTS/FIGURES/TABLES PAGES
%----------------------------------------------------------------------------------------

\tableofcontents % Prints the main table of contents

\listoffigures % Prints the list of figures

\listoftables % Prints the list of tables

\listofalgorithms
\addtocontents{loa}{\def\string\figurename{Algorithm}}

%----------------------------------------------------------------------------------------
%	ABBREVIATIONS
%----------------------------------------------------------------------------------------

\begin{abbreviations}{ll} % Include a list of abbreviations (a table of two columns)

\textbf{ADWIN} & Adaptive Windowing \\
\textbf{ANOVA} & Analysis of Variance\\
\textbf{AI} & Artificial Intelligence \\
\textbf{Bagging} & Bootstrap Aggregation \\
\textbf{CPU} & Central Processing Unit \\
\textbf{CSV} & Comma Separated Values\\
\textbf{DDM} & Drift Detection Method \\
\textbf{FHDDMS} & Fast Hoffding Drift Detection Method for evolving data Streams \\
\textbf{GB} & Gigabyte \\
\textbf{IoT} & Internet of Things \\
\textbf{kNN} & k Nearest Neighbours \\
\textbf{LB} & Leveraging Bagging \\
\textbf{LED} & Light Emitting Diode \\
\textbf{LESS-TWE} & Learning from Evolving Stream via Self-Training Windowing Ensembles \\
\textbf{ML} & Machine Learning \\
\textbf{MFHDDMS} & Modified Fast Hoffding Drift Detection Method for evolving data Streams \\
\textbf{MOA} & Massive Online Analysis \\
\textbf{NB} & Naive Bayes \\
\textbf{NN} & Neural Network \\
\textbf{OS} & Operating System \\
\textbf{RAM} & Random Access Memory \\
\textbf{SEA} & Streaming Ensemble Algorithm \\
\textbf{SGD} & Stochastic Gradient Descent \\
\textbf{SSD} & Solid State Drive \\
\textbf{SSL} & Semi-Supervised Learning \\
\textbf{TN} & True Negative \\
\textbf{TP} & True Positive \\
\textbf{UCI} & University of California, Irvine \\
\textbf{VFDT} & Very Fast Decision Tree \\
\textbf{WEKA} & Waikato Environment for Knowledge Analysis \\




\end{abbreviations}

%----------------------------------------------------------------------------------------
%	THESIS CONTENT - CHAPTERS
%----------------------------------------------------------------------------------------

\mainmatter % Begin numeric (1,2,3...) page numbering

\pagestyle{thesis} % Return the page headers back to the "thesis" style

% Include the chapters of the thesis as separate files from the Chapters folder
% Uncomment the lines as you write the chapters

% Chapter 1

\chapter{Introduction\label{chapter:introduction}} % Main chapter title

%----------------------------------------------------------------------------------------

% Define some commands to keep the formatting separated from the content 
\newcommand{\keyword}[1]{\textbf{#1}}
\newcommand{\tabhead}[1]{\textbf{#1}}
\newcommand{\code}[1]{\texttt{#1}}
\newcommand{\file}[1]{\texttt{\bfseries#1}}
\newcommand{\option}[1]{\texttt{\itshape#1}}

Analytical models have been studied and developed by researchers, with increasing intensity over the last two decades, to intelligently and computationally extract knowledge from data \cite{bifet2009data}. The quantity and size of data sets have grown exponentially over that time, requiring new algorithms to respect new constraints. Data streams are a latest data format that researchers are developing techniques for, and, are characterised by velocious and continuous flows of data \cite{krempl2014open}. This format requires algorithms to update their models to adapt to potential changes to the underlying concepts that the stream represent over time. Techniques have been developed to explicitly detect these changes to help models to adapt more quickly in order to stay relevant and accurate. Researchers have shown a continuing interest in the development of algorithms to extract knowledge from evolving streams \cite{gama2010knowledge, gama2014survey, ghesmoune2016state, KRAWCZYK2017132, krempl2014open, silva2013data, widmer1996learning}. Another current topic of interest is the development of ensembles, which are defined as an amalgamation of any number of analytical models. Ensembles are considered as one of the most promising research directions nowadays \cite{jain2000statistical, KRAWCZYK2017132, oza2008classifier, polikar2006ensemble, rokach2009taxonomy, wozniak2014survey}.

An abundance of algorithms have been proposed in the literature to use ensembles to learn from evolving streams, that do so either online (read instance-by-instance) or in chunks. These algorithms typically learn in batches by training new classifiers on each incoming chunk, and either use some weighting scheme or replacement strategy that relies on the use of timely and correctly labelled data. Alternatively, algorithms can learn online by using sliding windows of a variable, or fixed, size to summarise the data and appropriately update their models from them.

Drift detection methods have also mainly relied on the use of labelled data by detecting changes in the accuracy of a classifier over time. \textbf{On the other hand} drift detectors that are unsupervised, mostly rely on statistical tests \cite{dries2009adaptive, friedman1979multivariate, sheskin2003handbook, sobolewski2013concept}.

%state the purpose of the study. 
%allow readers to understand the background to the study, without needing to consult the literature themselves.
%To describe the historical development of the topic.
%To provide a context for the later discussion of the results.

%----------------------------------------------------------------------------------------

\section{Motivation}
A major issue with supervised and semi-supervised learning of data streams is late-arriving or missing class labels\textbf{. For example, a bank cannot know if a loan will default before several months have passed, if not years}. Assuming that correctly labelled data will always be available and timely is unreasonable, and, as such, supervised methods are not usually applicable in the real world. Therefore, real-world problems usually require the use of semi-supervised or unsupervised learning techniques. To the best of our knowledge, no research has been conducted to develop semi-supervised techniques to learn from evolving streams without clustering unlabelled data, which is computationally expensive \cite{krempl2014open}. Furthermore, semi-supervised drift detecting algorithms for streams \textbf{are more} applicable to real-world applications, but only one article has been published as of yet \cite{haque2015sand}.
%----------------------------------------------------------------------------------------

\section{Thesis Objective}
The purpose of this thesis is to contribute to narrow the gap in research as it pertains to semi-supervised learning of evolving streams, without clustering unlabelled data.

Therefore, we combined our contributions to introduce a framework, \textbf{LESS-TWE} (Learning from Evolving Streams via Self-Training Windowing Ensembles). Our framework employs self-training, a novel windowing technique exclusive to ensembles, a new weighted soft voting strategy, and an extension of \textbf{Fast Hoeffding Drift Detection Method for evolving Streams (FHDDMS)} to work without labelled data.

Research has not yet been conducted to study how selective self-training\textbf{, a semi-supervised learning (SSL) algorithm,} performs when removing the selective heuristic, thereby predicting labels for unlabelled data then training on it. \textbf{Selective self-training learns from originally unlabelled data that it annotates with labels it predicts for which it has high confidence in.} This presents one of the objectives for this thesis.

By introducing the novel windowing technique, we aim to investigate if savings relating to the execution time can be achieved while maintaining comparable predictive accuracy. Additionally, we can observe if delaying the training of some the classifiers in the ensemble affects concept drift detection.

Finally, we introduce a weighted soft voting scheme for ensembles, in conjunction with our extension to FHDDMS to detect drifts without relying on labelled data.

%----------------------------------------------------------------------------------------

\section{Thesis Organisation}
\textbf{This thesis is organised as follows.}

In chapter \ref{chapter:background_work} we review the background work surrounding data stream mining, ensemble learners, and concept drift detection.

Following the review, we will present the contributions made by this thesis to the literature in chapter \ref{chapter:contributions}.

As stated above, this will cover improvements to an existing concept drift algorithm to reduce its dependency on ground truth, improvements to a simple voting classifier to also further reduce its dependency on ground truth and finally a novel windowing technique that combines sliding and tumbling techniques.

In chapter \ref{chapter:experimental_design}, we will present the experimental design for testing our contributions and present the results and discuss them in chapter \ref{chapter:evaluation_discussion}.

Finally we conclude in chapter \ref{chapter:conclusion}.
% Chapter 3

\chapter{Background Work} % Main chapter title

\label{Chapter2} % For referencing the chapter elsewhere, use \ref{Chapter3} 

%----------------------------------------------------------------------------------------

% Bartosz Krawczyk  
% Chapter 3
\chapter{Contributions\label{chapter:contributions}}% For referencing the chapter elsewhere, use \ref{chapter:contributions} } % Main chapter title

In this chapter, we introduce our RGTDSMF methodology. We will describe our new weighting scheme and our novel windowing technique for ensembles. We will then cover our improvement to the voting classifier to reduce the dependency on ground truth during training. Next, we will explain how we extended FHDDM and FHDDMS to work without ground truth. Finally, we will cover all of our contributions by using a small toy example to explain how data is processed from a stream.

%----------------------------------------------------------------------------------------

\section{Voting Classifier: Weighting and Voting Schemes \label{section:new_voting_strategy}}

The reader may refer to section \ref{section:new_voting_strategy} for background on ensembles and voting classifiers.

A voting classifier was extracted from the mlxtend library\footnote{\url{http://rasbt.github.io/mlxtend/user_guide/classifier/EnsembleVoteClassifier/}}, developed by Sebastian Raschka. Mlxtend is an extension of the scikit-learn python machine learning library, to handle classification in a streaming setting.

The mlxtend voting classifier implements two voting strategies: the first being "soft" voting and the second being "hard" (majority) voting; again, refer to section \ref{section:new_voting_strategy} for their definitions.

We implemented two additional voting schemes similar to the "soft" scheme in that they require its classifiers to output a probability or confidence in the class label prediction. The first method makes use of a logistic function with a sigmoid curve to weigh the probability, centred around a  confidence level. Let $\alpha$ be the parameter that defines the confidence level, set to \textbf{\textit{$\alpha=50\%$}}. The second method, explained soon thereafter is equivalent to the function of the first method while being less computationally intensive.


Logistic functions are defined by the function in equation \ref{eq:logistic_function}
\begin{equation}
    f(x)=\frac{L}{1+e^{-k(x-x_0)}}\\ 
    \label{eq:logistic_function}
\end{equation}
where \textit{e} is the natural logarithm base, \textit{$x_0$} is the x-value of the sigmoid's midpoint, \textit{L} is the curve's maximum value, and \textit{k} is the steepness of the curve.

The logistic function that we initially selected had the following values for the variables listed above. $x_0=0.65$, $L=1$, and $k=14$. These values were selected based on the resulting graph for values of $x \in [0, 1]$ and $f(x) \in [0, 1]$. As figure \ref{graph:weights} illustrates, predictions with less than approximately forty percent probability are essentially treated as predictions with zero probability; predictions with under a seventy percent probability are diminished in importance, and those over seventy percent are boosted. Refer to table \ref{table:weight_probabilities} to see how the values change by five and ten percent increments. The reasoning motivating this choice is our intuition that we should put more trust in predictions with over seventy percent (70\%) probability, put less trust in predictions with a probability of under seventy percent  (70\%), and virtually none in predictions with a probability under forty percent (40\%).

\begin{figure}
  \includegraphics[width=\linewidth]{./images/chapter3/weight_graph}
\caption{\label{graph:weights}Voting classifier weighting functions\\\protect\blueline$x$\\
\protect\redline $\frac{1-tanh(3.5-7x)}{2}$}
\end{figure}

However, after extensive experimentation, of which the results are found in tables \ref{table:weighting_experimental_test} and \ref{table:weighting_experimental_test_all_datasets}, we found that setting \textbf{$x_0=0.50$} and \textbf{$k=14$} was more optimal than the values we initially selected through intuition for the $x_0$ and $k$ parameters.

The same test as above was repeated, but, over four separate data sets for the two best combinations of parameters and the one from our intuition ($k=14$, and $x_0 \in [0.5, 0.55, 0.65]$), and ran only once. Table \ref{table:weighting_experimental_test_all_datasets} shows, again, that $k=14, x_0=0.50$ seems to be the optimal set of parameters as it obtains the best results for three out of four data sets.

\begin{table}[]
\caption{\label{table:weighting_experimental_test}Experimental test of weighting function parameters}
\centering
\begin{tabular}{|c|c|c|c|c|}
\hline
\textbf{Parameters} & \textbf{Accuracy} & \textbf{$\kappa$} & \textbf{$\kappa_t$} & \textbf{$\kappa_m$} \\ \hhline{=====}
\textit{no weighting}&81.95&61.47&61.10&52.78 \\ \hline
10, 0.65&83.56&64.77&64.57&57.00 \\ \hline
14, 0.45&84.28&66.26&66.12&58.88 \\ \hline
\textbf{14, 0.50}&\textbf{84.77}&\textbf{67.34}&\textbf{67.17}&\textbf{60.15} \\ \hline
\textbf{14, 0.55}&\textbf{84.72}&\textbf{67.25}&\textbf{67.06}&\textbf{60.02} \\ \hline
14, 0.65&83.75&65.12&64.98&57.49 \\ \hline
14, 0.6&84.22&66.13&65.98&58.71 \\ \hline
14, 0.75&81.73&60.86&60.63&52.21 \\ \hline
14, 0.85&80.51&58.65&57.99&49.01 \\ \hline
5, 0.65&82.77&63.03&62.85&54.91 \\ \hline
7, 0.65&83.04&63.56&63.44&55.62 \\ \hline
\end{tabular}
\end{table}

\begin{table}
\caption{\label{table:weighting_experimental_test_all_datasets}Experimental test of weighting function parameters on 4 datasets}
\centering
\begin{tabular}{|c|c|c|c|c|c|}
\hline
\textbf{Dataset} & \textbf{Parameters} ($k=14$) & \textbf{Accuracy} & \textbf{$\kappa$} & \textbf{$\kappa_t$} & \textbf{$\kappa_m$} \\ \hhline{======}
& $x_0=0.5$    &    84.30    &    68.60    &    68.69    &    68.73\\ \cline{2-6}
sine1 & $x_0=0.55$    &    84.15    &    68.30    &    68.39    &    68.44\\ \cline{2-6}
 & $x_0=0.65$    &    83.79    &    67.59    &    67.67    &    67.72\\ \hhline{======}
 & $x_0=0.5$    &    79.95    &    59.90    &    60.07    &    60.12\\ \cline{2-6}
mixed & $x_0=0.55$    &    79.89    &    59.78    &    59.96    &    60.01\\ \cline{2-6}
 & $x_0=0.65$    &    79.60    &    59.20    &    59.38    &    59.43\\ \hhline{======}
 & $x_0=0.55$    &    79.72    &    59.45    &    59.44    &    59.15\\ \cline{2-6}
circles & $x_0=0.65$    &    79.43    &    58.87    &    58.87    &    58.57\\ \cline{2-6}
 & $x_0=0.5$    &    77.72    &    55.45    &    55.45    &    55.13\\ \hhline{======}
 & $x_0=0.5$    &    85.20    &    68.27    &    68.10    &    61.28\\ \cline{2-6}
SEA & $x_0=0.55$    &    84.25    &    66.25    &    66.05    &    58.80\\ \cline{2-6}
 & $x_0=0.65$    &    83.41    &    64.38    &    64.25    &    56.60\\ \hline
\end{tabular}
\end{table}

\begin{table}[]
\caption{\label{table:weight_probabilities}Probabilities after weighting}
\centering
\begin{tabular}{|c|c|c|c|c|c|c|c|c|c|c|c|}\\ \hhline{============}
\textbf{Probability (\%)} & 0-10&20&30&40&50&60&65&70&75&80-85&90-100 \\ \hline
\textbf{Weighted (\%)} & 0&1&6&20&50&80&89&94&97&99&100 \\ \hline
\end{tabular}
\end{table}

We must note that $\forall x \in [0,1], f(x) \in [0, \frac{1}{1+e^{-7}}\approx 0.9991]$ for the function in equation \ref{eq:logistic_function} , we could then divide the function by its value for the maximum value for $p_i(X)$ (which is 1) to ensure that our function covers all values in $[0,1]$, but this is unnecessary as all values will be be in the same range, and it adds another calculation.
Refer to figure \ref{graph:weights} for the plot of equation \ref{eq:logistic_function_params}.

\begin{equation}
\frac{1}{1+e^{-14(p_i(X)-0.50)}}
\label{eq:logistic_function_params}
\end{equation}

Our proposed voting strategy therefore computes the average of this logistic function using the prediction of each classifier for each class label. See the following equation:
\begin{equation}
\frac{1}{n}\sum_{i=1}^{n}\frac{1}{1+e^{-14(p_i(X)-0.50)}}\\ 
    \label{eq:logistic_sum}
\end{equation}
where $n$ is the number of classifiers in the voting ensemble, and $p_i(X)$ is the probability of classifier $i$ predicting that the tuple in question belongs to class X.

We propose another weighting equation to determine if it is possible to achieve similar results by using a different function that resembles the logistic sigmoid function presented above while being less computationally intensive. As Dr. LeCun states in \citep[10]{lecun2012efficient},  "hyperbolic tangent functions often converge faster than the standard logistic function".
Our hyperbolic tangent weighting function is seen in equation \ref{eq:tanh_weight_fn} and the sum is seen in equation \ref{eq:tanh_sum}. The plot of equation \ref{eq:tanh_weight_fn} is completely identical to the plot of \ref{eq:logistic_function}.
\begin{equation}
    f(x)=\frac{1-tanh(3.5-7x)}{2}
    \label{eq:tanh_weight_fn}
\end{equation}
\begin{equation}
\frac{1}{n}\sum_{i=1}^{n} \frac{1-tanh(3.5-7\times p_i(X))}{2}
    \label{eq:tanh_sum}
\end{equation}
The calculation of averages presented in equations \ref{eq:logistic_sum} and \ref{eq:tanh_sum} are calculated for each class label. The class label with the highest average is thereafter selected as the winner in the vote. We want to test the hypothesis that weighting the predictions by an exaggeration of their probability will increase the accuracy of our voting classifier.

In order to confirm that the new weighting function was more computationally efficient, we compared the total running time to evaluate each weighting function ten million times on a random value in [0, 1]. The benchmark showed, as seen in table \ref{table:weight_benchmark}, that the logistic function takes almost 3.5 times longer to compute than the hyperbolic tangent function, which will, therefore, be used for the remainder of this thesis.

\begin{table}[]
\centering
\caption{\label{table:weight_benchmark}Weighting function benchmark results}
\begin{tabular}{|c|c|}
\hline
\textbf{Sigmoid function} & \textbf{Hyperbolic tangent} \\ \hhline{==}
1441.05 ns/class/loop & 407.40 ns/class/loop \\ \hline
\end{tabular}
\end{table}

Algorithm \ref{alg:new_voting_scheme} shows the implementation of this new voting scheme using the hyperbolic tangent function. Figure \ref{graph:weights} shows a plot with the two new proposed weighting functions and compare them to the value of an unweighted prediction represented by $f(x) =x$, for  $x \in [0,1]$. This figure shows how much a value is boosted or reduced compared to its original value

Given that the hyperbolic tangent function selected is equivalent to the logistic sigmoid function, it is not necessary to test its parameters experimentally as they give the same output. Figure \ref{fig:boxplots_params} shows the boxplots of table \ref{table:weighting_experimental_test}. The whiskers correspond to the minimum and maximum values over 10 runs with each parameter. The dotted diamond corresponds to the standard deviation, the dotted line in the box corresponds to the mean, while the solid line corresponds to the median.

\begin{figure}
  \includegraphics[width=\linewidth]{./images/chapter3/boxplots_params}
\caption{\label{fig:boxplots_params}Boxplots of experimental test for logistic function parameters}
\end{figure}


\begin{algorithm}
    \caption{\label{alg:new_voting_scheme}tanh weighting scheme for voting classifier}
    \Fn{voting\_classifier.predict(X)}{
        \tcc{The probabilities are stored in a $1\times2$ matrix (using binary classification for simplicity) where each column represents a class, and each row represents a sub-classifier}
        weighted\_probabilities = []\;
        \ForEach{$classifier \in voting.classifiers$}{
            array = classifier.predict()\;
            $w\_p$ = [weight(array[0]), weight(array[1])]\;
            weighted\_probabilities.append($w\_p$)\;
        }
        \tcp{map sub-classifier probabilities to single probability}
        avg = weighted\_probabilities.avg\_over\_columns()\;
        prediction, prediction\_probability = max(avg), class\_max\_value(avg)\;
        \Return{prediction, prediction\_probability}\;
    }
    \Fn{weight(p)}{
        \Return $\frac{1-tanh(3.5-7p)}{2}$\;
    }
\end{algorithm}



%----------------------------------------------------------------------------------------

\section{Hybrid sliding-tumbling windows\label{section:hybrid-windows}}
In \cite{KRAWCZYK2017132}, B. Krawczyk mentions that one of the many issues in data stream mining is execution time, in the sense that our algorithm must learn faster than tuples can arrive. In our case, we aim to determine if we can delay training of some classifiers in our ensemble, and see how it affects execution time and classifying performance, as well as drift detection performance.

We propose the following algorithm, implemented in algorithm \ref{alg:sliding_tumbling_windows}, titled \textit{Sliding-Tumbling windows for training the ensemble}.
Let the number of tuples (single tuple or chunk) used in the interleaved test-then-train loop iterations be \textit{number\_of\_tuples}, and let \textit{number\_of\_classifiers} be the number of classifiers in the ensemble. The ensemble will have a window size of \textit{number\_of\_tuples}*\textit{number\_of\_classifiers}. At every iteration of the interleaved test-then-train loop, we will append the new tuples to the ensemble's window and train a single classifier in the ensemble on that window. For the next \textit{number\_of\_classifiers - 1} iterations, we will train the remaining \textit{number\_of\_classifiers - 1} classifiers in the ensemble. We do this so that from the point of view of the ensemble, we are training using sliding windows. However, from the point of each classifier in the ensemble, we are training them using tumbling windows.

While not used for the same purpose, the same sliding batches were used to improve CDC-Stream in \citep{d2016fine}. Figure \ref{fig:sliding_tumbling_windows}, taken from our colleague's thesis, shows how the algorithm works for three (3) classifiers in the ensemble with a batch size of one (1) and a window size of three (3). Each classifier will only learn from the same coloured batch; meaning that at time \textit{t}, only a single classifier has enough tuples to learn from, but the others will learn at time \textit{t+1} and finally at time \textit{t+2}. Each classifier will be learning from what essentially is a tumbling window, from their point of view, just not all from the same one, or at the same time. 

To clarify the example, at time $t_1$, our ensemble will receive only one instance, add it to the sliding window and train classifier $c_1$ on the window. At time $t_2$, another instance will be added to the window and the ensemble will train classifier $c_2$ on the new window containing two tuples. At time $t_3$, the ensemble will train classifier $c_3$ on the first purple chunk (the window now has its maximum of 3 instances). At time $t_4$, $c_2$ will learn from the first blue chunk. And at time $t_5$, $c_3$ will learn from the first green chunk. This process loops indefinitely. 

The motivation for this technique is to determine if we can spend less execution time training the classifiers. We also want to investigate how progressively delaying training of some of the classifiers in the ensemble affects concept drift detection and classification performance while also hopefully reducing execution time.

\begin{algorithm}
\KwResult{at least one classifier in the ensemble was trained}
\tcp{voting\_ensemble stores a classifier list, its count, and, the index of the current classifier to train}
\Fn{ensemble.train(X, y)}{    
        classifier\_to\_train = classifier\_list[index]\;
        index = (index + 1) modulo (number\_of\_classifiers)\;
        classifier\_to\_train.partial\_train(X, y)\;
}
\caption{Sliding-Tumbling windows for training the ensemble\label{alg:sliding_tumbling_windows}}
\end{algorithm}

\begin{figure}
  \includegraphics[width=\linewidth]{./images/chapter3/sliding_tumbling_windows}
  \caption{Sliding Tumbling windows \citep{d2016fine}.}
  \label{fig:sliding_tumbling_windows}
\end{figure}



%----------------------------------------------------------------------------------------

\section{Improvements to voting classifier to reduce dependency on ground truth\label{section:vc_reduce_gt}}

The interleaved test-then-train methodology in a data streaming setting has a rather significant flaw if we consider a real-life scenario: we are assuming that we obtain the ground truth immediately after testing. This means that we assume that the ground truth is always available in a fraction of a second after testing our models. For the vast majority of cases, this approach is not realistic, again, in a real-world setting.

Therefore, we wanted to determine if we could re-use the idea behind self-training in an offline setting to reduce our dependency on the ground truth in the online streaming setting for the interleaved test-then-train method. We propose an approach that is, as previously stated, similar to self-training in that we use the classifier's prediction, as opposed to correctly labelled instances, when training.

However, using only the predictions to train our model in an online setting is a recipe for disaster. The known cons to using self-training in an offline setting are that it can reinforce classification errors; it is, therefore, logical for us to conclude that we will encounter the same risks in porting this idea to a streaming setting.

Given the restriction of limiting reinforcing misclassification errors, we want to determine at what ratio of predictions to ground-truth our voting ensemble's accuracy would decline and by how much.

The algorithm behind this idea is very straightforward: it consists in duplicating the ground truth array and replacing at random a particular fraction of values with the actual prediction from the classifier, if there was no drift detected right before. See algorithm \ref{alg:self-training} to see the pseudocode (function \textit{swap ground truth with predictions}).

\begin{algorithm}
\While{stream.has\_more\_instances()}{
    X, y = stream.get\_next\_tuples(number\_of\_instances\_to\_fetch)\;
    predictions, probabilities = voting\_ensemble.predict(X)\;
    drift\_detected = voting\_ensemble.detect\_drift(predictions, probabilities)\;
    \If{$percentage \neq 100$ \textbf{and} drift not detected}{
        y = swap\_ground\_truth\_with\_predictions(y, predictions, percentage)\;
    }
 voting\_ensemble.train(X, y)\;
}

\tcp{This algorithm only shows the steps required to modify the ground truth array}
\Fn{swap\_ground\_truth\_with\_predictions(y, predictions, percentage)}{
    \For{$index=0\ \textbf{;}\ index < length(y)\ \textbf{;}\ index += 1$}{
        \If{random\_number\_between(0, 100) > percentage}{
            y[index] = predictions[index]\;
        }
    }
    \Return{y}\;
}
\caption{\label{alg:self-training}Online self-training}
\end{algorithm}

%----------------------------------------------------------------------------------------

\section{Extending FHDDM/S to function without labelled data}

Additionally, this thesis builds upon our colleague's, Dr. Pesaranghader, work described in section \ref{section:fhddm/s}

Along the same lines of section \ref{section:vc_reduce_gt}, the drift detection algorithm proposed by my colleague Dr. Pesaranghader in \cite{pesaranghader2016fast}, \textit{FHDDMS}, relies on the immediate knowledge of ground truth. The drift detection mechanism in FHDDMS relies on storing, in a sliding window, whether or not the classifier accurately predicted the class. This method only applies to a select few domains where correctly labelled data become available almost instantly after the tuples arrive, and in all synthetic data streams of course. For the vast majority of domains, this method is not applicable.

It is for that reason that we have set out to study if FHDDM/S is still able to detect drifts when completely removing its dependency on any knowledge of the ground truth in an online streaming setting by replacing the booleans, indicating that the classifier in/correctly predicted a class instance, stored in the sliding window by another variable.

In order to do so, we have opted to extend FHDDM and FHDDMS such that the sliding window now stores one of three values: the unweighted probability of the classifier's predicted class instance for the winning vote; the weighted probability of the classifier's predicted class instance for the winning vote; or a boolean indicating if the classifier's prediction corresponds to the ensemble's winning vote.

We implemented several approaches:
\begin{itemize}
\item one drift detector per classifier in the ensemble, storing the unweighted probabilities,
\item a single drift detector for the ensemble, storing the average of the unweighted predictions for the winning vote,
\item one drift detector per classifier in the ensemble, storing a boolean indicating if the classifier's prediction corresponds to the ensemble's winning vote.
\end{itemize}


After extensive experimentation, we were able to determine that when it comes to detecting drifts, the unweighted probabilities were no more useful than the weighted values as the difference in performance between the two is negligible. A SEA dataset was generated with 10\% noise with one-hundred thousand instances, with four concepts and three abrupt concept drifts at every twenty-five thousand instances. And the $CIRCLES$, $MIXED$ and $SINE1$ datasets were used (refer to section \ref{section:performance_measures}). When drifts were detected, all of the classifiers in the ensemble were completely reset, and one drift detector for the entire ensemble was used. In order to determine whether to use unweighted or weighted probabilities for the drift detector, we looked at the accuracy and kappa statistics of the entire stream (covered in section \ref{eq:kappa}). Table \ref{table:drift_use_weighting_experimental_test} shows these results. The values in the table are averages over ten runs.

\begin{table}[]
\caption{\label{table:drift_use_weighting_experimental_test}Testing whether to use weighted probabilities for detecting drifts}
\centering
\begin{tabular}{|c|c|c|c|c|c|}
\hline
\textbf{Dataset} & \textbf{Weighted/Unweighted} & \textbf{Accuracy} & \textbf{$\kappa$} & \textbf{$\kappa_t$} & \textbf{$\kappa_m$} \\ \hhline{======}
SEA&\checkmark&84.53&66.77&66.66&59.53\\ \cline{2-6}
 &$\times$&84.42&66.51&66.42&59.23\\ \hhline{======}
circles&$\times$&78.35&56.71&56.70&56.39\\ \cline{2-6}
 &\checkmark&78.25&56.51&56.50&56.19\\ \hhline{======}
mixed&\checkmark&80.00&60.00&60.18&60.23\\ \cline{2-6}
 &$\times$&79.97&59.94&60.11&60.16\\ \hhline{======}
sine1&\checkmark&84.35&68.71&68.79&68.84\\ \cline{2-6}
 &$\times$&84.30&68.60&68.68&68.73\\ \hline
\end{tabular}
\end{table}

Figure \ref{fig:boxplot_params_use_w} shows the results of the experimental test from table \ref{table:drift_use_weighting_experimental_test} using boxplots. As in the table, we can see that the weighted predictions cause performance to increase ever so slightly. In the case of the $CIRCLES$ dataset, the unweighted probabilities seem to perform better than the weighted ones.

\begin{figure}
  \includegraphics[width=\linewidth]{./images/chapter3/boxplot_params_use_w}
\caption{\label{fig:boxplot_params_use_w}Boxplots showing performance over 4 datasets (using un/weighted probabilities for drift detection)}
\end{figure}

Our modified drift detectors using the probabilities will be called MFHDDM/S (M for modified).
Algorithm \ref{alg:pfhddm} shows the implementation of MFHDDM. MFHDDMS is different from MFHDDM only in that it keeps track of an additional shorter sliding window. No changes were necessary to implement FHDDM/S with the different boolean, as Dr. Pesaranghader's currently works with boolean values.

\begin{algorithm}
\caption{Modified Fast Hoeffding Drift Detection Method (MFHDDM)\label{alg:pfhddm}}
\Fn{init(window\_size, delta, use\_probability)}{
    (n, $\delta$, p) = (window\_size, delta, use\_probability)\;
    $\epsilon_d = \sqrt{\frac{1}{2n}ln\frac{1}{\delta}}$\;
    reset()\;
}

\Fn{reset()}{
    w=[]\;
    $\mu^m=0$\;
}

\Fn{detect(p)}{
    \If{w.size = n}{
        w.tail.drop()\;
    }
    w.push(p)\;
    \eIf{$w.size < n$}{
        return False\;
    }{
        \eIf{use\_probability}{
            $\mu^t=w.average()$\;
        }{
            $\mu^t=\frac{w.count(True)}{w.size()}$\;
        }
        \If{$\mu^m < \mu^t$}{
            $\mu^m =\mu^t$\;
        }
        $\Delta\mu = \mu^m - \mu^t$\;
        \eIf{$\Delta\mu \ge \epsilon_d$}{
            reset()\;
            \Return True\;
        }{
            \Return False\;
        }
    }
}
\end{algorithm}

%----------------------------------------------------------------------------------------

\section{A toy example}

In this section, we will explain how an instance from the stream is processed from beginning to end. The reader should refer to algorithm \ref{alg:pipeline_pseudocode} and figure \ref{fig:not_yet_uploaded}.

The first step in the algorithm is pre-training the algorithm so that it can start with a decent model of the data. Therefore a parameter is used to dictate the number of instances that the ensemble is to train on before starting the interleaved test-then-train loop, where the entirety of the work of this thesis takes part in.

Next, the interleaved test-then-train loop begins. For this toy example, we will use a batch size of one. In this batch (or chunk), there will be a single instance, which we will refer to as $X$, and its true class value which will be referred to as $y$.

The ensemble will first be tasked with predicting the class value, denoted $\hat{y}$, of $X$. In order to do so, the ensemble will require each classifier it contains to assign a probability that $X$ belongs to a class, for each possible value that the class can take. For example, in the case of a binary classification problem, $y \in [0,1]$. So in our example, each classifier needs to predict $P(X, y=0)$, and $P(X, y=1)$ given its model of the data. The ensemble will then keep a copy of these predictions and apply a weighting function to the original values. The weighting function, as previously stated, reduces values in $[0,0.7[$ and increases values in $]0.7, 1]$. Finally, the ensemble will average the probabilities for each class class value (again, for binary classification). It will do the same for the weighted probabilities. The ensemble will now have four values: $p_0, p_1 = \frac{1}{n}\sum_{i=1}^{n} P_i(X, y=Y)\ \forall\ Y$,  $w_0,w_1=\frac{1}{n}\sum_{i=1}^{n} w(P_i(X, y=Y))\ \forall\ Y$, where $n$ is the number of classifiers in the ensemble, $w(x)$ is the weighting function seen in section \ref{section:new_voting_strategy}, $P_i(X, y=0)$ is the probability that classifier $i$ assigns $X$ to class $0$. $Y$ is the set of possible class values.
The maximum of $p_0$ and $p_1$ will determine the class the ensemble predicts for $X$. If $p_0$ is the maximum value, then $\hat{y}=0$, and in the opposite case where $p_1$ is the maximum value then $\hat{y}=1$.

The next step in the interleaved test-then-train loop is dedicated to drift detection. When drifts are detected, we want to keep a sliding window of the previously seen instances (the size of this window is the same as the window size used for the training step) for retraining the classifiers. In order to detect drifts, a modified version of FHDDMS is used. The average probability for the winning class, which we mentioned earlier as being copied, is passed to the drift detector. Until the sliding windows within the drift detector are full, the detector will not run. There are two sliding windows: one short to detect abrupt drifts, and a longer one to detect gradual concept drifts. The sliding window will, therefore, keep track, as discussed in section \ref{section:new_voting_strategy}, of either the probabilities or the boolean indicating whether a classifier predicted the "winning" vote. In the case of the probabilities, the Hoeffding bound is used to detect if the average probability drifts too far from the maximum seen average probability. In the case of the boolean values, a drift is detected, using the Hoeffding bound again, when the classifier in question predicts less often on average according to the "winning" vote.
When a drift is detected, the drift detector's windows are emptied, the classifiers in the ensemble are either all reset or a subset of them are reset. 
If a drift is detected, the sliding window of previously seen instances is then used to retrain all of the classifiers in the ensemble.

If no drift is detected, there is a chance (set by a parameter) that the $y$ value (the real class value of $X$) will be replaced by $\hat y$ (the predicted class value of $X$).

The final step in the interleaved test-then-train loop is to train the ensemble on the $<X, y>$ tuple. In order to do so, a hybrid sliding-tumbling window (as seen in section \ref{section:hybrid-windows}) is used within the ensemble to train the classifiers contained, again, within it. In this case, since $<X, y>$ is the first instance seen by the ensemble during the interleaved test-then-train loop (as in this is the first iteration of the prequential evaluation loop) it is added to the hybrid window. This hybrid window has a size of the batch size (in this case 1) times the number of classifiers in the ensemble (in this case let us say 3). 
Only one classifier in the ensemble is trained on the hybrid window's tuples. In order to do so, an index is kept inside the ensemble. This index points to the classifier to train and is incremented at each $ensemble.train()$ call, and a modulus is applied on the index to keep it within the range of classifiers in the ensemble.
So classifier $c_0$ is selected to train on the hybrid window, containing a single tuple at the moment.

At this point, the first iteration of the interleaved test-then-train loop has terminated, and the framework computes the global metrics required (accuracy, kappa statistics, etc.) as well as over a sliding window of the last 200 instances.

\begin{algorithm}
\caption{Data processing pipeline\label{alg:pipeline_pseudocode}}
\Fn{evaluate\_prequential(ensemble, pretrain\_size, batch\_size, window\_size)}{
    ensemble.train(pretrain\_size)\;
    window = fifo\_queue(size = window\_size)\;
    \While{stream.has\_more\_samples()}{
        X, y = stream.next\_sample(batch\_size)\;
        window.add([X,y])\;

        prediction, probabilities, weighted\_probabilities = ensemble.test(X)\;
        
        drift\_detected = drift\_detection(prediction\ \textbf{or}\ probabilities\ \textbf{or}\ weighted\_probabilities)\;

        \If{random() > ground\_truth\_percentage \textbf{and}\ not\ drift\_detected}{
            replace\_y(y, prediction)\;
        }
        
        \If{drift\_detected}{
            ensemble.reset(window)\;
        }
        
        ensemble.train(X, y)\;
    }
}

\Fn{ensemble.train(X, y)}{
    \tcp{instance variables $clfs$ stores the classifiers in ensemble \& $mod$ stores the current index 
    of the classifier to train}
    ensemble.clfs[ensemble.mod].train(X, y)\;
    ensemble.mod = (ensemble.mod + 1) modulus (length(ensemble.clfs))\;
}

\Fn{ensemble.test(X)}{
    p, w\_p = [], []\;
    \For{$classifier \in ensemble.classifiers$}{
        prob = classifier.predict()\;
        p.add(prob), w\_p.add([weight(prob[0]), weight(prob[1]))\;
    }
    \tcp{map subclassifier probabilities to single probability}
    avg, w\_avg = p.avg(), w\_p.avg()\;
    prediction = class\_of\_max(w\_avg)\;
    \Return prediction, avg, w\_avg\;
}

\Fn{weight(prediction\_probability)}{
    \Return $\frac{1-tanh(3.5-7\times prediction\_probability)}{2}$\;
}

\Fn{drift\_detection(value, use\_probability)}{
    window.push(value)\;
    \If{$window.size > max\_window\_size$}{
        window.tail.drop()\;
        \eIf{use\_probability}{
            $\mu^t=window.average()$\;
        }{
            $\mu^t=\frac{window.count(True)}{window.size()}$\;
        }
        \If{$\mu^m < \mu^t$}{
            $\mu^m =\mu^t$\;
        }
        $\Delta\mu = \mu^m - \mu^t$\;
        \If{$\Delta\mu \ge \epsilon\_d$}{
            \Return True\ \textbf{and} reset();
        }
    }
    \Return False\;
}
\end{algorithm}

%----------------------------------------------------------------------------------------

\section{Summary}
Throughout this chapter, we have introduced our RGTDSMF methodology. The first was to propose a new weighting function for our ensemble classifier. The second was a novel windowing technique for classifiers within an ensemble that delays their training. The third was investigating how implementing our ensemble to self-train on a portion of the training data could influence its performance. The fourth and final contribution was to extend FHDDM/S to operate without any knowledge of the ground truth. We have described the components of our methodology and their algorithms.


\chapter{Experimental Design\label{chapter:experimental_design}} % Main chapter title
In this chapter, we describe our experimental design.

The data sets used for our analysis in the upcoming chapter are \textbf{SEA, CIRCLES, SINE1 and MIXED, which all containing noise, and either abrupt or gradual concept drifts.}

The estimation technique we use is prequential evaluation, also known as interleaved test-then-train, which consists of infinitely executing a loop where a classifier first predicts labels for new data (without its label), then adapts its model for said data, with the correct label.

The performance measures that we use are the execution time, measured in seconds, as well as the $\kappa$-temporal statistic to evaluate a classifier's predictive performance, also called $\kappa^+$ or $\kappa_t$. This $\kappa$ statistic compares our classifier to a no-change classifier and takes into account temporal dependence in the data.

Using the mean values for the entire stream for both of the metrics mentioned above, we use statistical tests to determine whether or not the differences observed are statistically significant and not due to simple coincidence. When comparing two classifiers across multiple data sets, we use the Wilcoxon test, and when comparing more than two classifiers, we use the Friedman test, coupled with the post-hoc Nemenyi test.

We conduct the following experiments:
\begin{itemize}
\item  examine the impact of each parameter value on the mean of each metric,
\item rank the results of each parameter combination in order to establish any trend regarding parameter values across the metrics,
\item compare the top ranking parameter combinations to the state of the art.
\end{itemize}

%----------------------------------------------------------------------------------------

\section{Software and Hardware specifications}
In order to ensure that our experiments are reproducible, we specify the specifications of the hardware and software used to run them. All experiments were run on a MacBook Pro model \textit{11,4}. This machine was running macOS 10.14.4 on a quad-core Intel i7 2.2GHz processor (\textbf{3.4GHz} TurboBoost, 8 virtual cores), with a 256 GB SSD, and \textbf{16GB of RAM}. During the course of the experiments, the laptop was plugged in and fully charged.

Python 3.7.3 was installed, with the \textbf{following dependencies}:
\begin{itemize}
\item \textit{sortedcontainers} v2.0.5
\item \textit{numpy} v1.15.3
\item \textit{scipy} v1.1.0
\item \textit{scikit-learn} v0.20.0
\item \textit{pandas} v0.23.4
\item \textit{mlxtend} v0.13.0
\item \textit{scikit-multiflow} (latest commit from master rebased onto our branch) \textit{\textbf{\#f18a433}}
\end{itemize}

\section{Scikit-multiflow}
Scikit-multiflow \cite{skmultiflow} is a Python framework, that complements scikit-learn, an offline, batch learning Python library. As it currently stands, scikit-multiflow implements stream generators, machine learning algorithms, drift detection algorithms and evaluation methods for data streams. The authors describe scikit-multiflow as "an open-source framework for multi-output / multi-label and stream data mining", and takes its inspiration from MOA \cite{bifet2010moa} and MEKA \cite{read2016meka}. The former is the most popular data stream mining framework, implemented in Java, which includes a collection of machine learning algorithms, and evaluation tools. The latter is another project implemented in Java, but for multi-label learning and evaluation.

We selected this framework as it is backed by Bifet, implemented in Python: a popular language among data scientists, open-source, and very recent.

\section{Data sets\label{section:datasets}}
All but one of the synthetic data sets ($CIRCLES$, $SINE1$, and $MIXED$) were generated by my colleague for his paper \cite{pesaranghader2016fast} where he proposed FHDDMS, the drift detection algorithm. These data sets were generated with ten percent (10\%) noise to test the robustness of his drift detection algorithm against noisy data streams. The data sets contain one hundred thousand rows belonging to one of two classes. While the specific data sets we use were generated by Pesaranghader, the synthetic data sets can be traced back to \cite{10.1007/3-540-59286-5_74} and were further used in the following papers \cite{baena2006early,bifet2007learning,gama2004learning,nishida2007detecting,olorunnimbe2015intelligent}. % Given that these data sets are synthetic with drifts beginning at a known location, we can also determine the drift detection delay as well as true and false positive rates.

\subsection{$CIRCLES$}
As stated by Gama et al. in \cite{gama2004learning}, this data set is composed of two relevant numerical attributes: $x$ and $y$, which are uniformly distributed in [0, 1]. There are four different concepts in this data set, each representing whether or not a point is within a circle given $x$, and $y>$ coordinates for its centre and its radius $r_c$. This data set contains gradual concept drifts that occur at every twenty-five thousand (25 000) instances. The four pairs of $<(x,y), r_c>$ defining each concept are given in table \ref{table:circle_concepts}.

\begin{table}[]
\centering
\caption{\label{table:circle_concepts}$CIRCLES$ data set concepts}
\begin{tabular}{|c|c|c|c|c|}
\hline
center & (0.2, 0.5) & (0.4, 0.5) & (0.6, 0.5) & (0.8, 0.5) \\ \hline
radius & 0.15       & 0.2        & 0.25       & 0.3        \\ \hline
\end{tabular}
\end{table}

\subsection{$SINE1$}
As stated by Gama et al. in \cite{gama2004learning}, this data set contains abrupt concept drifts, with noise-free examples. It has only two relevant numerical attributes, for which the values are uniformly distributed in [0, 1]. Before the concept drift, all instances for values below the curve $y = sin(x)$ are classified as \textbf{positive}. Then, after the concept drift, the rule is reversed; therefore the values below the curve become \textbf{negative}. The drifts were generated at every twenty thousand (20 000) instances.

\subsection{$MIXED$}
As stated by Gama et al. in \cite{gama2004learning}, this data set contains abrupt concept drifts and uses four relevant attributes. Two of which are boolean, let them be $v$ and $w$; and the other two attributes are numerical, in [0, 1]. Instances belong to the positive class if two of three conditions are met: $v$ is true, $w$ is true, $y < 0.5 + 0.3 \times sin(3\pi x)$. For each concept drift, the conditions are reversed, meaning that if the conditions are met, it will be a positive instance, then after the drift, it will be a negative instance. The abrupt concept drifts occur at every twenty thousand (20 000) instances.

\subsection{Streaming Ensemble Algorithm generator}
First described in \cite{street2001streaming} by Street and Kim, the Streaming Ensemble Algorithm (SEA) generates streams with abrupt concept drift. It is composed of three numerical attributes of values in [0, 10], and only the first two attributes are relevant. For each instance, the class is determined by checking if the sum of the two relevant attributes passes a threshold value. Let $f_1$ and $f_2$ be the two numerical relevant attributes, and $\theta$ the threshold. An instance belongs to class \textit{one} if $f_1 + f_2 \leq \theta$. As in Street's paper, our stream has four concepts, with the threshold values for each being 8, 9, 7 and 9.5. We generate streams of one hundred thousand instances, from zero to twenty percent noise, in ten percent increments ($\{0; 10; 20\%\}$). Drifts, therefore, occur at every twenty-five thousand instances.

\section{Estimation techniques}
In an offline setting, with static data, the most common evaluation method used is called cross-validation. However, given how little time a model has to learn from each instance due to the velocity of data streams and the risk of concept drift, cross-validation is not suited for an online setting. The two following techniques are, though.

\subsection{Holdout}
Two distinct data sets are required for this technique: a training data set to train a model, and another to test it.  Cross-validation is typically used in offline static data mining, but is too computationally heavy and/or can be too time-consuming in a streaming setting, and therefore the validation step is skipped, and performance is measured against a single holdout set \cite{bifet2009data}.
 
This technique is most useful when the training and testing sets have already been defined, as it makes comparing results from different studies possible.
In order to track performance, the model is evaluated periodically, but not too frequently to avoid negatively impacting performance.

The holdout data set can be sourced from new tuples arriving from the data stream, and can also be added later to the training set in order to optimise instance use.

It should be sufficient to safely use a single static holdout set if we make the assumption that there is no concept drift \textbf{\cite{bifet2009data}}.

\subsection{Interleaved test-then-train, or prequential}
Another evaluation method is prequential evaluation, also known as the interleaved test-then-train method. The evaluation method consists of first testing the classifier on a given set of instances, then training the classifier on that same set. The evaluation strategy, therefore, ensures that the model has not previously seen testing tuples, and no holdout testing set is necessary, and the accuracy of the model is therefore incrementally updated. This method also allows us to use all of the data for both testing and training. As more data is tested than in the holdout method, each instance used to assess the accuracy and performance of the model weighs less than it would have in a smaller holdout test set \textbf{\cite{bifet2009data}}.

All experiments are done using the interleaved test-then-train evaluation technique.

\section{Performance measures\label{section:performance_measures}}

\subsection{Accuracy \& Confusion Matrix}
Let us assume that we are dealing with a binary classification problem (with two classes) with a \textbf{P}ositive and \textbf{N}egative class.

A confusion matrix keeps track of the correctness of our classifying model by using the following four measurements \cite[77-79]{japkowicz2011evaluating}.

\begin{itemize}
\item False Positive (FP): number of instances incorrectly classified as positive
\item False Negative (FN): number of instances incorrectly classified as negative
\item True Positive (TP): number of instances correctly classified as positive
\item True Negative (TN): number of instances correctly classified as negative
\end{itemize}

It should then be clear that all of the positive class instances (P) are in the combined groups of FN and TP, and that all of the negative class instances (N) are in the remaining FP and TN groups.

A confusion matrix helps us to visualise these measurements by presenting them in a matrix (two-by-two grid in the case of a binary classification problem). Each column represents all of the instances for an actual class, whereas each row represents the instances predicted for a given class. Table \ref{table:confusion_matrix} shows a confusion matrix.

\begin{table}[]
\centering
\caption{Confusion Matrix\label{table:confusion_matrix}}
\begin{tabular}{|c|c|c|}
\hline
                   & Actual Positive & Actual Negative \\ \hline
Predicted Positive & TP              & FN              \\ \hline
Predicted Negative & FP              & TN              \\ \hline
\end{tabular}
\end{table}

Given our four measurements above, we can calculate the following metrics:
\begin{itemize}
\item The \textit{sensitivity} and \textit{specificity} (or \textit{recall}) indicate the completeness for a given class of instances retrieved that belong to that class \cite[96]{japkowicz2011evaluating}; in other words the percentage of correctly classified instances of a given class that were found overall.
\begin{equation}
\frac{TP}{TP+FN}=\frac{TP}{P} \text{ and } \frac{TN}{TN+FP}=\frac{TN}{N}
\end{equation}
\item The \textit{accuracy}, while not a reliable metric, indicates the number of correct predictions over all cases to be predicted \cite[86]{japkowicz2011evaluating}.
\begin{equation}
sensitivity\times\frac{P}{P+N}\times specificity\times\frac{N}{P+N} = \frac{TP+TN}{P+N}
\end{equation}
\item \textit{Precision} indicates the exactness of predictions for a given class \cite[99]{japkowicz2011evaluating}. In other words, the percentage of instances the classifier predicted as positive that were actually positive,\begin{equation}\frac{TP}{TP+FP}\end{equation}
\item \textit{F-measure}, which is the harmonic mean of precision and recall \cite[103]{japkowicz2011evaluating}. \begin{equation}2\times\frac{precision\times recall}{precision+recall}\end{equation}
\end{itemize}

\subsection{Kappa ($\kappa$) statistics\label{section:kappa_stats}}
\subsubsection{$\kappa$ statistic}
Bifet and Frank argue in \cite{bifet2010sentiment} that prequential accuracy is not well suited for classifying unbalanced data in a stream whereas the Kappa ($\kappa$) statistic proposed by Cohen in \cite{cohen1960coefficient} is better adapted when dealing with changing class distributions. They therefore proposed a sliding-window kappa as defined by equation \ref{eq:kappa}
\begin{equation}
\kappa=\frac{p_0-p_c}{1-p_c}
\label{eq:kappa}
\end{equation}where $p_0$ is the classifier's prequential accuracy and $p_c$ is the probability that a chance (or no-change) classifier makes a correct prediction. Refer to \cite{bifet2010sentiment} for the equations used to calculate $p_0$ and $p_c$.

Values of $\kappa$ are in [0, 1]. If the classifier always classifies instances correctly, then $p_0=1$ and $\kappa=1$; if the classifier correctly classifies instances as often as the chance classifier, then $p_0 = p_c$ and $\kappa=0$; and if the chance classifier is always right, then $p_c=1$ and $\kappa=1$.

\subsubsection{$\kappa^+$ statistic\label{section:kappa_t}}
The $\kappa^+$ statistic is an improvement upon the regular $\kappa$ statistic that takes into account temporal dependence in order to better evaluate the performance of a given classifier. When there is no temporal dependence in the data and that classes are balanced, then $\kappa^+$ is equal to $\kappa$. While we could solely rely on $\kappa^+$, the authors however recommend using both $\kappa$ and $\kappa^+$ statistics in order to obtain a more thorough evaluation.
$\kappa^+$ is defined in equation \ref{eq:kappa_plus}
\begin{equation}
\label{eq:kappa_plus}
\kappa^+=\frac{p_0-p'_e}{1-p'_e}
\end{equation}where $p_0$ is the classifier's prequential accuracy and $p'_e$ is the accuracy of a no-change classifier. The no-change classifier predicts that the next class label will be the same as the last seen class label \cite{bifet2015efficient}. Just like the original $\kappa$ statistic, $\kappa^+\in [0, 1]$. Refer to \cite{DBLP:conf/pkdd/2013-1} for the equations relating to calculating $p_0$ and $p'_e$. $\kappa^+$ is also sometimes referred to as $\kappa$-temporal ($\kappa_t$) \cite{vzliobaite2015evaluation}.

\subsubsection{$\kappa_m$ statistic}
The $\kappa_m$ statistic is another improvement upon the regular to indicate whether a classifier is performing better than a majority-class classifier \cite{bifet2015efficient}. Its definition is given in equation \ref{eq:kappa_m}.

\begin{equation}
    \label{eq:kappa_m}
    \kappa_m = \frac{p_0-p_m}{1-p_m}
\end{equation}where $p_0$ is the classifier's prequential accuracy and $p_m$ is the prequential accuracy of a majority-class classifier. If the classifier always predicts correctly, then $\kappa_m=1$. If the classifier predicts correctly as often as the majority-class one, then $\kappa_m=0$.

\subsection{Testing for Statistical Significance}
While the measures presented in \ref{section:performance_measures} are useful for evaluating the performance of our classifiers, it is insufficient to rely solely on them to fully evaluate performance differences between classifiers. Statistical significance tests are used to determine whether or not the differences that were observed are statistically significant and not due to simple coincidence \cite{japkowicz2011evaluating}.

Popular choices for statistical tests in the machine learning community for comparing multiple classifiers across multiple cases are the Analysis of Variance (ANOVA) \cite{fisher1956statistical} and the Friedman test \cite{friedman1937use}. 

The ANOVA test assumes a normal data distribution whereas the Friedman test does not. It is for this reason, coupled with the fact that the Friedman test is also non-parametric, that we have chosen the latter as our choice for a statistical significance test. However, the Friedman test is not recommended when only comparing two algorithms on multiple domains \cite[355]{flach2012ml}; for that reason, we will be using the Wilcoxon test \cite{wilcoxon1945individual} in those scenarios.

\subsubsection{The Friedman test}
The test ranks each algorithm separately for each data set and computes a test statistic using the variation within the ranks and the variation within the error variation. The Friedman statistic $\chi^2_F$ is defined in equation \ref{eq:friedman_statistic}, taken from \cite{japkowicz2011evaluating};

\begin{equation}
\label{eq:friedman_statistic}
\chi^2_F = \frac{SS_{Total}}{SS_{Error}}=\bigg[\frac{12}{n\times k\times(k+1)}\times\sum_{j=1}^k(R_.j)^2\bigg]-3\times n \times (k+1)
\end{equation}where $n$ is the number of data sets, $k$ is the number of algorithms considered, and $R$ simply denotes ranking. And finally $(k-1)$ is known as the degrees of freedom.

The Friedman statistic ($\chi^2_F$) is then looked up in the $\chi^2$ distribution table to obtain a \textit{probability value} (p-value), signifying $P(\chi^2_{k-1} \geq \chi^2_F )$. This p-value is widely used in null-hypothesis testing by measuring it against a threshold value called a significance level. A lower p-value means a higher statistical significance. Traditionally, the significance level is either 0.1 or 0.5, and denoted as $\alpha$. The null hypothesis is accepted if the p-value is greater than $\alpha$; otherwise, $H_1$ is accepted, and the null hypothesis is rejected.

When the null hypothesis is rejected following a Friedman test, post hoc Nemenyi tests are recommended by Japkowicz and Shah in \cite{japkowicz2011evaluating}, as well as Flach in \cite[355]{flach2012ml}.

\subsubsection{Wilcoxon's Signed-Rank test}
This test is a non-parametric version of the matched-paired \textit{t}-test. It is used to test if two paired samples come from the same distribution, by measuring the difference in their mean ranks.

The test procedure is as follows \cite[233-235]{japkowicz2011evaluating}, \cite[354]{flach2012ml}:

$N$ will be set as the sample size, meaning that there are $2N$ data points, and since data are paired, $[x_{1, i}, x_{2, i}]$ represent the measurements for pairs where $i\in [1 ; N]$.

We test the following null hypothesis.

$H_0 $: difference between the pairs follows a symmetric distribution around zero
\newline$H_1$: does not follow a symmetric distribution around zero

\begin{itemize}
\item Calculate $d_i = (x_{1, i} - x_{2, i}) \forall i\in[1;N]$.
\item Pairs for which the difference is zero are excluded, let $N_r$ be the reduced sample size.
\item Rank the $N_r$ pairs by their absolute difference in increasing order, let $R_i$ denote rank. In the case of ties, assign the pairs with the average of their ranks.
\item Two sum of ranks are then calculated: 
\newline$W_{s 1}=\sum_{i=1}^{N_r} I(d_{i}>0) \operatorname{rank}(d_{i}),$ 
\newline$W_{s 2}=\sum_{i=1}^{N_r} I(d_{i}<0) \operatorname{rank}(d_{i})$.
\item Calculate statistic $T_{wilcox}=\min(W_{s 1}, W_{s 2})$.
\item For smaller values of $N$, which is the case in our thesis, look up tabulated critical values of $T$ (according to N and statistical significance level) and reject the null hypothesis if the statistic is inferior to the tabulated value.
\end{itemize}



\subsubsection{Post-hoc test: the Nemenyi test}
The Nemenyi test \cite{nemenyi1962distribution} is used in order to determine which algorithms compared in the Friedman test actually differ. It computes a statistic, $q$, as defined in equation \ref{eq:nemenyi_q} for a pair of classifiers $f_{j1}$ and $f_{j2}$. $q$ gives us the difference between ranks of the two algorithms, and can be also expressed as a critical difference (CD) \cite[356]{flach2012ml}. In order compare the rankings, the Nemenyi test computes the mean rank for a given classifier using equation \ref{eq:nemenyi_mean_rank} where $R_{ij}$ is the rank of classifier $f_j$ on data set $S_i$ \cite[256-257]{japkowicz2011evaluating}.
\begin{equation}
\label{eq:nemenyi_q}
\overline{R}_{.j}=\frac{1}{n}\sum_{i=1}^nR_{ij}
\end{equation}\begin{equation}
\label{eq:nemenyi_mean_rank}
q=\frac{\overline{R}_{.j1}-\overline{R}_{.j2}}{\sqrt{\frac{k(k+1)}{6n}}}
\end{equation}

\section{Experimental Setup}
For these experiments, the following classifiers were used: Gaussian Naive Bayes (G\_NB), Leveraging Bagging, Stochastic Gradient Descent (SGD), Multinomial Naive Bayes (M\_NB), and our Voting Ensemble composed of three sub-classifiers. The three classifiers inside the voting ensemble are the SGD, M\_NB, and G\_NB. Finally, we also used a no-change classifier as well as a majority-class classifier as our baselines.

Readers familiar with classification algorithms \cite{viktor2015course} might know that some advantages of NB are its scalability, high accuracy, ability to use prior knowledge, and the fact that it has comparable performance to decision trees and neural networks. NB is incrementally more confident, and easy to implement. However, the downsides are its significant compute costs, and that using conditional dependencies reduces accuracy due to the real dependency between attributes.
SGDs have generally high predictive accuracy, generalise well meaning that they are robust and suitable for noisy domains, and good for incremental learning. However, they are known to have a slow training time, may fail to converge, and output a black box model which means that results cannot be interpreted.
Leveraging bagging \cite{bifet2010leveraging} is an improvement over the Online Bagging technique of Oza and Russel. They introduce more randomisation to the online bagging algorithm to achieve a more accurate classifier. However, the authors noted that subagging, half subagging, and bagging with-out replacement ran faster but were marginally less accurate.

Unless stated otherwise, our default parameters for the experiments are listed in table \ref{table:default_exp_parameters} and the default parameters for the classifiers are listed in table \ref{table:default_clf_parameters}. These values were obtained through extensive experimentation.

Finally, each experiment is run on five different examples each sourced from three synthetic data sets (5 examples of $SINE1$, another 5 of $CIRCLES$, etc.) and three streams generated by the SEA generator with levels of noise in increments of ten percent (from 0\% to 20\%), for a grand total of eighteen (18) streams of one hundred thousand (100000) instances.

\begin{table}[]
\centering
\caption{\label{table:default_exp_parameters}Default parameters for the experiments}
\begin{tabular}{|l|l|}
\hline
\multicolumn{1}{|c|}{\textbf{Parameter}} & \multicolumn{1}{c|}{\textbf{Value}} \\ \hline \hhline{==}
Classifier & Ensemble Voting Classifier \\ \hline
Pretrain size & 1 000 \\ \hline
Max samples & 100 000 \\ \hline
Batch size & 25 \\ \hline
Window size & 75 \\ \hline
Window type & sliding tumbling hybrid \\ \hline
Ground truth \% & 100 \\ \hline
Drift reset type & none \\ \hline
\end{tabular}
\end{table}

\begin{table}[]
\centering
\caption{\label{table:default_clf_parameters}Our default parameters for all classifiers}
\begin{tabular}{|l|l|}
\hline
\multicolumn{1}{|c|}{\textbf{Classifier}} & \multicolumn{1}{c|}{\textbf{Parameters}} \\ \hline \hhline{==}
Leveraging Bagging & \begin{tabular}[c]{@{}l@{}}base\_estimator=HoeffdingTree()\\ n\_estimators=10\\ w=6\\ delta=0.002\\ leverage\_algorithm=leveraging\_bag\\ random\_state=None\end{tabular} \\ \hline
Hoeffding Tree & \begin{tabular}[c]{@{}l@{}}max\_byte\_size=33.5MB\\split\_criterion=information gain\\split\_confidence=0.0000001\\binary\_split=False\\remove\_poor\_atts=False\\no\_preprune=False\\leaf\_prediction=Naive Bayes Adaptive\end{tabular} \\ \hline
Gaussian NB & \begin{tabular}[c]{@{}l@{}}priors=None\\ var\_smoothing=1e-9\end{tabular} \\ \hline
Multinomial NB & \begin{tabular}[c]{@{}l@{}}alpha=1.0\\ fit\_prior=True\\ class\_prior=None\end{tabular} \\ \hline
Stochastic Gradient Descent & \begin{tabular}[c]{@{}l@{}}loss=log\\ penalty=l2\\ alpha=0.0001\\ l1\_ratio=0.15\\ fit\_intercept=True\\ max\_iter=1000\\ tol=1e-3\\ shuffle=True\\ epsilon=0.1\\ n\_jobs=-1\\ random\_state=None\\ learning\_rate=optimal\\ eta0=0.0\\ power\_t=0.5\\ early\_stopping=False\\ validation\_fraction=0.1\\ n\_iter\_no\_change=5\\ average=False\\ n\_iter=None\end{tabular} \\ \hline
\end{tabular}
\end{table}

Our voting ensemble takes seven (7) different parameters, some of which are conditional upon the value of others. The parameters and the values they can take are shown in table \ref{table:ensemble_params}. Some combinations of parameters are not allowed such as combining boolean voting with non-boolean drift detector content, or using probability voting with weighted probability drift content. In the worst case, when counting the illegal combinations, there are $4\times2\times3\times5\times4\times3\times2=1880$ combinations. There are 1110 permitted parameter combinations.

\begin{table}[]
\caption{Voting ensemble parameters\label{table:ensemble_params}}
\begin{tabular}{|l|l|}
\hline
\textbf{Parameter} & \textbf{Values} \\ \hline \hhline{==}
Voting type & boolean, probability, avg. w. probability, w. avg. probability \\ \hline
Window type & sliding, hybrid \\ \hline
Chunk size & 25, 75, 100 \\ \hline
Ground truth & 60\%, 70\%, 80\%, 90\%, 100\% \\ \hline
Drift reset type & no drift detection, blind, partial, complete \\ \hline
Drift content & boolean, probability, weighted probability \\ \hline
Drift detector count & 1 or many \\ \hline
\end{tabular}
\end{table}

We will run our algorithm for each permitted combination of the parameters stated above, and will measure  $\kappa_t$ and the execution time for each of the eighteen data sets. We will then evaluate how each parameter affects the execution time and $\kappa_t$.

We will test the null hypothesis that all values of a given parameter will perform similarly in regards to a given measure when we set all other parameters. We will repeat this experiment over all combinations of the "other" parameters.

For parameters that have only two values, we will use the Wilcoxon test to determine if the values of the parameters lead to statistically significant differences in the metrics measured. We will also use the internal intermediate results of the Wilcoxon test to rank the parameter values, again depending on the metric measured.

Otherwise, for parameters that can take more than two values, we will use the Friedman test; and when the Friedman test rejects the null hypothesis, we will use a post-hoc Nemenyi test to determine which pairs of values lead to statistically significant differences in the measured metrics.

We will be comparing the values of each parameter across all permitted parameter combinations. We will then be able to \textbf{assess} if a parameter value is often ranked better and if it seems to influence the measures even when changing the "other" set parameters. In other words, we \textbf{aim} to \textbf{determine} if a parameter is likely to frequently affect our measured metrics no matter of the other parameters.

After we \textbf{assess} the impact of the parameter values on the measured metrics, we will use the ranking algorithm from the post-hoc Nemenyi test to rank each parameter combination to determine the top ranking combination for each measured metric and over both metrics.

Once we obtain the parameter combinations that lead to the best results, we will compare them to the state of the art algorithm (leveraging bagging). We will, also, be comparing these results with no-change and majority-class classifiers (trained with 100\% labelled examples and sliding windows). The leveraging bagging classifier is implemented with a built-in ADWIN drift detector. As we will be comparing at least 3 algorithms, we will employ the Friedman test in conjunction with the post-hoc Nemenyi test where appropriate to determine which pairs of algorithms differ.

In summary, we:
\begin{enumerate}
\item compare each value of one parameter (while setting other parameters to a given value), multiple times and each time changing the other parameter combination to \textbf{assess} the impact of that one parameter value on the measured metrics
\item rank the results of each parameter combination
\item use those rankings to compare the top parameter combinations to the state of the art
\end{enumerate}

In the following chapter, we present the results of our experiments, analyse these findings and discuss their significance. 
% Chapter 5

\chapter{Experimental Evaluation and Discussion} % Main chapter title

Recall from the previous chapter that in order to evaluate our contributions, we will need to compare how our algorithm performs against 4 different data sets. We will be using the $\kappa_t$ metric, as well as the execution time. Finally, we will also be considering the percentage of labelled data is used to train our ensemble.
{In this chapter, we will be investigating how each parameter influences each measured metric using omnibus Wilcoxon and Friedman tests, and in the case of the latter, post-ho\textbf{}c Nemenyi tests to further confirm which pairs of algorithms differ in performance. Next, we will compare all of the parameter combinations together by ranking them by each metric. This is done to find any trends that lead to better performance. Finally, we will compare our approach to the state of the art}

\label{Chapter5} % For referencing the chapter elsewhere, use \ref{Chapter3} 

\begin{figure}
  \includegraphics[width=\linewidth]{./images/kappa_vs_time}
\caption{\label{fig:kappa_vs_time}$\kappa_t$ in relation to time, across all parameter combinations}
\end{figure}
Figure \ref{fig:kappa_vs_time} shows that as the execution time increases, $\kappa_t$ stays roughly constant. The majority of the changes seem to be due to the various parameter combinations. This is a good sign as it suggests that the predictive accuracy of our model is at most very loosely tied to the execution time of its algorithm.

\section{Investigating how each parameter influences each metric}

\subsection{Wilcoxon tests}

\subsubsection{Drift Detector Count}

\begin{table}[]
\centering
\caption{\label{table:wilcoxon_significant}Statistically significant percentage of parameter combinations found via the Wilcoxon test}
\begin{tabular}{|l|l|l|l|l|l|}
\hline
\textbf{Parameter} & \textbf{Measure} & \textbf{0.05} & \textbf{0.01} & \textbf{0.001} & \textbf{Total} \\ \hline \hhline{======}
\multirow{2}{*}{Drift detector count} & execution time & 5\% & 6\% & 82\% & 93\% \\ \cline{2-6} 
 & $\kappa_t$ & 4\% & 1\% & 1\% & 7\% \\ \hline\hhline{======}
\multirow{2}{*}{Window type} & execution time & 1\% & 6\% & 89\% & 97\% \\ \cline{2-6} 
 & $\kappa_t$ & 7\% & 8\% & 44\% & 60\% \\ \hline
\end{tabular}
\end{table}

As we can see from table \ref{table:wilcoxon_significant}, the parameter \textit{Drift Detector Count} seems to heavily influence the execution time of the algorithm, regardless of the other parameter values. When it comes to the $\kappa_t$ metric, only 7\% of combinations proved to have statistically significant differences in predictive performance. 

\begin{table}[]
\centering
\caption{\label{table:wilcoxon_drift_detector_count}Statistically significant percentage of parameter combinations by parameter value for Drift Detector Count found via the Wilcoxon test}
\begin{tabular}{|l|l|l|l|l|l|l|l|l|}
\hline
\multirow{3}{*}{\textbf{Measure}} & \multicolumn{4}{l|}{\textbf{1 for ensemble}} & \multicolumn{4}{l|}{\textbf{1 per classifier}} \\ \cline{2-9} 
 & \multicolumn{2}{l|}{\textbf{significant}} & \multicolumn{2}{l|}{\textbf{insignificant}} & \multicolumn{2}{l|}{\textbf{significant}} & \multicolumn{2}{l|}{\textbf{insignificant}} \\ \cline{2-9} 
 & \textbf{count} & \textbf{\%} & \textbf{count} & \textbf{\%} & \textbf{count} & \textbf{\%} & \textbf{count} & \% \\ \hline \hhline{=========}
execution time & 447 & 99\% & 22 & 73\% & 4 & 0\% & 8 & 26\% \\ \hline
$\kappa_t$ & 19 & 59\% & 206 & 48\% & 13 & 40\% & 217 & 51\% \\ \hline
\end{tabular}
\end{table}


\begin{figure}
\centering
  \includegraphics[width=\linewidth]{./images/wilcoxon_drift_detector_count_pie}
\caption{\label{fig:wilcoxon_drift_detector_count_pie}Pie chart illustrating table \ref{table:wilcoxon_significant}}
\end{figure}



When we dig deeper into the \textit{Drift Detector Count} parameter, as we can see from table \ref{table:wilcoxon_drift_detector_count} and figure \ref{fig:wilcoxon_drift_detector_count_pie}, for the execution time metric, the parameter value \textit{1 for ensemble} is evidently the best choice as it ranks best across 92\% of parameter combinations, and best across 99\% of statistically significant different results. For the $\kappa_t$ metric, it is not as clear cut; both \textit{1 for ensemble} and \textit{1 per classifier} rank best among about 50\% of the time. 

The results we found lead us to believe that choosing the \textit{1 for ensemble} value for the \textit{Drift Detector Count} parameter is very beneficial in reducing the execution time of the algorithm. While this finding might not hold across all data streams examined, it is likely that choosing this value for the parameter will decrease execution time significantly. Logically, it makes perfect sense that choosing \textit{1 for ensemble} over \textit{1 per classifier} leads to lower execution time because the implementation of the former is such that it performs only a fraction of the operations of the latter. Additionally, we also found that none of the parameter values consistently outranked the others in predictive accuracy, and we therefore cannot say if choosing a particular parameter value will result in better values for the $\kappa_t$ metric. We will later be looking at the raw values to see if a particular drift detector count leads to better predictive accuracy.

\subsubsection{Window Type}
As for the \textit{Window Type} parameter, $97\%$ of parameter combinations show a significant statistical difference in the execution time depending on the value of the window type. This indicates that the parameter value heavily influences the execution time of the algorithm, independently of other parameter values. When it comes to the $\kappa_t$ metric, over half (60\%) of combinations proved to show a statistically significant difference in predictive performance. This suggests that the \textit{Window Type} parameter could have some non-negligible influence over the $\kappa_t$ metric.

\begin{table}[]
\centering
\caption{\label{table:wilcoxon_window_type}Statistically significant percentage of parameter combinations by parameter value for Window Type found via the Wilcoxon test}
\begin{tabular}{|l|l|l|l|l|l|l|l|l|}
\hline
\multirow{3}{*}{\textbf{Measure}} & \multicolumn{4}{l|}{\textbf{Hybrid}} & \multicolumn{4}{l|}{\textbf{Sliding}} \\ \cline{2-9} 
 & \multicolumn{2}{l|}{\textbf{significant}} & \multicolumn{2}{l|}{\textbf{insignificant}} & \multicolumn{2}{l|}{\textbf{significant}} & \multicolumn{2}{l|}{\textbf{insignificant}} \\ \cline{2-9} 
 & \textbf{count} & \textbf{\%} & \textbf{count} & \textbf{\%} & \textbf{count} & \textbf{\%} & \textbf{count} & \% \\ \hline \hhline{=========}
execution time & 514 & 99\% & 6 & 54\% & 5 & 0\% & 5 & 45\% \\ \hline
$\kappa_t$ & 9 & 2\% & 86 & 40\% & 319 & 97\% & 125 & 59\% \\ \hline
\end{tabular}
\end{table}


\begin{figure}
  \includegraphics[width=\linewidth]{./images/wilcoxon_window_type_pie}
\caption{\label{fig:wilcoxon_window_type_pie}Pie chart illustrating table \ref{table:wilcoxon_window_type}}
\end{figure}


When we dig deeper into the \textit{Window Type} parameter, as we can see from table \ref{table:wilcoxon_window_type}  and figure \ref{fig:wilcoxon_window_type_pie}, for the execution time metric, \textit{hybrid} is evidently the best choice as it ranks best across 98\% of parameter combinations. For the $\kappa_t$ metric, it is just as obvious as \textit{sliding} ranks best across about 82\% of the parameter combinations. 

The \textit{Window Type} parameter ranking results suggest that \textit{hybrid windowing} has a significant impact on the execution time of the algorithm, independently of other parameters values. This suggests that choosing the \textit{hybrid windowing} value for the \textit{Window Type} parameter could be very beneficial in reducing the execution time of the algorithm. Again, this is exactly as expected: the implementation of \textit{hybrid windowing} is such that each sub-classifier in the ensemble only trains on each instance once, whereas the \textit{sliding windowing} technique is implemented such that each sub-classifier in the ensemble trains on each instance at least once. This, logically, leads to fewer operations, and therefore a reduction in execution time for the algorithm. This should hold true across all data streams, and therefore we recommend anyone who chooses to run this algorithm with the intent of reducing the execution time to choose the \textit{hybrid windowing} parameter value.
However, we also found that \textit{sliding windowing} outperforms \textit{hybrid windowing} across most parameter combinations, with or without a significant statistical difference. Again, using the explanation above for the reduction in execution time, since each sub-classifier training using \textit{sliding windowing} trains on each instance multiple times, it is logical that it is better able to fit the data instance in its model. 
This means that the algorithm presents a trade-off between execution time and predictive accuracy when considering the \textit{Window Type} parameter. Whoever runs the algorithm must choose which metric they value more, and choose a windowing type accordingly.

As other parameters can take more than two values, we must use the Friedman test in combination with the post-hoc Nemenyi test in order to test our null hypotheses.

\subsection{Post-hoc Nemenyi tests}

\begin{table}[]
\centering
\caption{\label{table:nemenyi_significant_breakdown}Percentage of parameter combinations that showed statistically significant differences from the post-hoc Nemenyi test}
\begin{tabular}{|l|l|c|l|c|l|}
\hline
\textbf{Parameter} & \textbf{Measure} & \textbf{Significant} & \textbf{\%} & \textbf{Insignificant} & \textbf{\%} \\ \hline \hhline{======}
\multirow{2}{*}{Batch size} & $\kappa_t$ & $\frac{1}{330}$ & $0.3\%$ & $\frac{229}{330}$ & $99.70\%$ \\ \cline{2-6} 
 & execution time & $\frac{287}{330}$ & $86.97\%$ & $\frac{43}{330}$ & $13.03\%$ \\ \hline \hhline{======}
\multirow{2}{*}{Drift reset type} & $\kappa_t$ & $\frac{1}{330}$ & $0.30\%$ & $\frac{329}{330}$ & $99.70\%$ \\ \cline{2-6} 
 & execution time & $\frac{64}{330}$ & $19.39\%$ & $\frac{266}{330}$ & $80.61\%$ \\ \hline \hhline{======}
\multirow{2}{*}{Ground truth} & $\kappa_t$ & $\frac{57}{198}$ & $28.79\%$ & $\frac{141}{198}$ & $71.21\%$ \\ \cline{2-6} 
 & execution time & $\frac{168}{198}$ & $84.85\%$ & $\frac{30}{198}$ & $15.15\%$ \\ \hline \hhline{======}
\multirow{2}{*}{Voting type} & $\kappa_t$ & $\frac{0}{180}$ & $0\%$ & $\frac{180}{180}$ & $100.00\%$ \\ \cline{2-6} 
 & execution time & $\frac{154}{180}$ & $85.56\%$ & $\frac{26}{180}$ & $14.44\%$ \\ \hline
\end{tabular}
\end{table}

\begin{table}[]
\centering
\caption{\label{table:nemenyi_significant_breakdown_aggregate}Number of different ranking and statistically significant parameter combinations from table \ref{table:nemenyi_significant_breakdown}}
\begin{tabular}{|l|l|c|c|}
\hline
\textbf{Parameter} & \textbf{Measure} & \textbf{Significant} & \textbf{Insignificant} \\ \hline \hhline{====}
\multirow{2}{*}{Batch size} & $\kappa_t$ & $ 1 \rightarrow 1 $ & $ 229 \rightarrow 6 $ \\ \cline{2-4} 
 & execution time & $ 287 \rightarrow 2 $ & $ 43 \rightarrow 2 $ \\ \hline \hhline{====}
\multirow{2}{*}{Drift reset type} & $\kappa_t$ & $ 1 \rightarrow 1 $ & $ 229 \rightarrow 6 $ \\ \cline{2-4} 
 & execution time & $ 64 \rightarrow 6 $ & $ 266 \rightarrow 6 $ \\ \hline \hhline{====}
\multirow{2}{*}{Ground truth} & $\kappa_t$ & $ 57 \rightarrow 3 $ & $ 141 \rightarrow 6 $ \\ \cline{2-4} 
 & execution time & $ 168 \rightarrow 22 $ & $ 30 \rightarrow 16 $ \\ \hline \hhline{====}
\multirow{2}{*}{Voting type} & $\kappa_t$ & $ 0 \rightarrow 0 $ & $ 180 \rightarrow 4 $ \\ \cline{2-4} 
 & execution time & $ 154 \rightarrow 2 $ & $ 26 \rightarrow 3 $ \\ \hline
\end{tabular}
\end{table}

\subsubsection{Batch Size}

As we can see from table \ref{table:nemenyi_significant_breakdown}, only 1 combination of parameters among 330 were significantly statistically different when considering the $\kappa_t$ metric for the batch size parameter, which is a negligible amount. As for the execution time metric, we can see that almost $87\%$ of parameter combinations proved to show a significant statistical difference. This confirms our expectations that changing only the batch size has a large impact on the execution time of the algorithm but no significant impact on its predictive accuracy.
The first row of table \ref{table:nemenyi_significant_breakdown_aggregate} indicates that there are 7 variations of rankings of batch size rankings for the $\kappa_t$ metric; the second line indicates that there are 2 variations of rankings and pairs of parameters that were significantly statistically different and 2 variations of ranks for batch sizes that were insignificant. These values are seen in table \ref{table:batch_size_rankings} and figure \ref{fig:batch_size_rankings_pie}.

\begin{table}[]
\centering
\caption{\label{table:batch_size_rankings}Rankings for batch size and parameter combination counts}
\begin{tabular}{|l|c|l|c|c|}
\hline
\textbf{Metric} & \textbf{Stat. Sig.} & \textbf{Ranks} & \textbf{Stat. Sig. values} & \textbf{\%} \\ \hline \hhline{=====}
\multirow{7}{*}{$\kappa_t$} & \checkmark & 25 / 75 / 100 & 25 / 100 & 0.30\% \\ \cline{2-5} 
 & \multirow{6}{*}{$\times$} & 100 / 25 / 75 & \multirow{6}{*}{} & 16.36\% \\ \cline{3-3} \cline{5-5} 
 &  & 100 / 75 / 25 &  & 10.30\% \\ \cline{3-3} \cline{5-5} 
 &  & 25 / 100 / 75 &  & 16.67\% \\ \cline{3-3} \cline{5-5} 
 &  & 25 / 75 / 100 &  & 27.58\% \\ \cline{3-3} \cline{5-5} 
 &  & 75 / 100 / 25 &  & 10.00\% \\ \cline{3-3} \cline{5-5} 
 &  & 75 / 25 / 100 &  & 18.79\% \\ \hline \hhline{=====}
\multirow{4}{*}{Execution time} & \multirow{2}{*}{\checkmark} & 100 / 75 / 25 & 100 / 25 & 71.52\% \\ \cline{3-5} 
 &  & 75 / 100 / 25 & 75 / 25 & 15.45\% \\ \cline{2-5} 
 & \multirow{2}{*}{$\times$} & 100 / 75 / 25 & \multirow{2}{*}{} & 6.67\% \\ \cline{3-3} \cline{5-5} 
 &  & 75 / 100 / 25 &  & 6.36\% \\ \hline
\end{tabular}
\end{table}

\begin{figure}
  \includegraphics[width=\linewidth]{./images/batch_size_rankings_pie}
\caption{\label{fig:batch_size_rankings_pie}Pie chart illustrating table \ref{table:batch_size_rankings}}
\end{figure}

Let us examine the pie charts illustrating the rankings of the \textit{Batch size} parameter values.
We will first examine the results for the execution time. At first glance, we notice that \textit{100} clearly ranks first over the majority of parameter combinations, followed by \textit{75} over a minority of combinations.
For the $\kappa_t$ metric, the rankings are less clear. \textit{25} seems to rank first more frequently (across $45\%$ of parameter combinations), whereas \textit{75} and \textit{100} rank first across $26\%$ of parameter combinations each. These rankings do not tell us how much the difference is across parameter combinations. By this, we mean that while a parameter value may rank first across a larger percentage of parameter combinations, it may actually perform worse than another value over a smaller percentage of parameter combinations.
In any case, the results indicate that there is another trade-off to be made in regards to the batch size parameter between predictive accuracy and execution time. Again, we expected these results as a batch size of 100 allows the ensemble to learn from more examples at once, which reduces the quantity of operations that would have to be repeated were we to use a batch size of 25 for example, where each operation down the line would be run 4 times instead of only 1 time. As for predictive accuracy, \textit{25} has a slight advantage over the other parameter values possibly because there is a better chance that a drift wreaks havoc on a sub-classifier's ability to train on a batch of \textit{75} or \textit{100} than on a batch of \textit{25}. For a batch of \textit{25}, the sub-classifiers can be reset sooner, and the next batch may be easier to model.

\subsubsection{Drift Reset Type}

Let us now investigate the Drift Reset Type parameter. As we can see from table \ref{table:nemenyi_significant_breakdown}, only a single out of 330 parameter combinations showed a significant statistical difference in parameter values when considering the $\kappa_t$ metric, whereas about $20\%$ of parameter values showed a statistical significant difference in parameter values when considering the execution time metric. This suggests that the value of the drift reset type parameter does not have a significant impact over other values on the measured metrics.
Table \ref{table:nemenyi_significant_breakdown_aggregate}, lists the number of different rankings found in table \ref{table:drift_reset_type_rankings} over the measured metrics, depending on whether results showed a statistical significant difference.

Let us examine table \ref{table:drift_reset_type_rankings} or better yet, the pie charts seen in figure \ref{drift_reset_type_rankings_pie} illustrating the rankings of the \textit{Drift Reset Type} parameter values.
Again, we will first examine the results for the execution time. At first glance, we notice that \textit{blind resets} clearly ranks first the least over the parameter combinations. \textit{Reset all} ranks first across half of the parameter combinations and \textit{partial resets} over about a third of them. This is a good sign, and expected. Since sub-classifiers must be reset more frequently, they must be completely re-trained on data, which also prevents them from learning from a larger set of older data. This might be good for streams with very frequent drifts, but those where they appear infrequently, blindly resetting the classifier prevents it from remembering possibly useful historical data.

For the $\kappa_t$ metric, the rankings are not as clear. Again, we must remember that these rankings do not tell us the difference across parameter combinations, meaning that while a parameter value may rank first across a larger percentage of parameter combinations, it may perform less well than another value over a smaller percentage of parameter combinations.
In any case, the results indicate that each parameter value ranks first across a third of parameter combinations. This does not allow us to conclude much, other than blind resets may perform just as well as resetting every sub-classifier or only a minority of them. It could be that the sub-classifiers are not very apt at learning from the data, or that resetting the sub-classifiers is not very effective. Another reason could be attributed to the way partial drift resets work, in that each sub-classifier has a chance of being reset, meaning that there is a chance that the offending sub-classifier that is not properly adapting to the concept drift is not reset. A more thorough investigation is unfortunately outside of the scope of this work, due to time constraints. It could also be that the predictive accuracy can not improve much after a drift due to the limitations of the sub-classifiers to model the data effectively. Either way, this is truly unfortunate, as it suggests that changing how often and how many sub-classifiers are reset does not seem to affect the $\kappa_t$ metric.
These results indicate that blind resets perform almost as well as our other reset strategies, which may suggest that our drift reset strategy needs further investigation.

\begin{table}[]
\centering
\caption{\label{table:drift_reset_type_rankings}Rankings for drift reset type and parameter combination counts}
\begin{tabular}{|l|c|l|c|c|}
\hline
\textbf{Metric} & \textbf{Stat. Sig.} & \textbf{Ranks} & \textbf{Stat. Sig. values} & \textbf{\%} \\ \hline \hhline{=====}
\multirow{7}{*}{$\kappa_t$} & \checkmark & BLIND / ALL / PARTIAL & BLIND / PARTIAL & 0.30\% \\ \cline{2-5}
 & \multirow{6}{*}{$\times$} & ALL / BLIND / PARTIAL & \multirow{6}{*}{} & 23.64\% \\ \cline{3-3} \cline{5-5} 
 &  & ALL / PARTIAL / BLIND &  & 13.64\% \\ \cline{3-3} \cline{5-5} 
 &  & BLIND / ALL / PARTIAL &  & 15.45\% \\ \cline{3-3} \cline{5-5} 
 &  & BLIND / PARTIAL / ALL &  & 17.88\% \\ \cline{3-3} \cline{5-5} 
 &  & PARTIAL / ALL / BLIND &  & 14.24\% \\ \cline{3-3} \cline{5-5} 
 &  & PARTIAL / BLIND / ALL &  & 14.85\% \\ \hline \hhline{=====}
\multirow{12}{2cm}{Execution time} & \multirow{6}{*}{\checkmark} & ALL / PARTIAL / BLIND & ALL / BLIND & 6.97\% \\ \cline{3-5} 
 &  & ALL / BLIND / PARTIAL & ALL / PARTIAL & 0.61\% \\ \cline{3-5} 
 &  & BLIND / PARTIAL / ALL & BLIND / ALL & 0.91\% \\ \cline{3-5} 
 &  & BLIND / ALL / PARTIAL & BLIND / PARTIAL & 0.91\% \\ \cline{3-5} 
 &  & PARTIAL / BLIND / ALL & PARTIAL / ALL & 1.52\% \\ \cline{3-5} 
 &  & PARTIAL / ALL / BLIND & PARTIAL / BLIND & 8.48\% \\ \cline{2-5} 
 & \multirow{6}{*}{$\times$} & ALL / BLIND / PARTIAL & \multirow{6}{*}{} & 6.36\% \\ \cline{3-3} \cline{5-5} 
 &  & ALL / PARTIAL / BLIND &  & 36.36\% \\ \cline{3-3} \cline{5-5} 
 &  & BLIND / ALL / PARTIAL &  & 7.58\% \\ \cline{3-3} \cline{5-5} 
 &  & BLIND / PARTIAL / ALL &  & 5.45\% \\ \cline{3-3} \cline{5-5} 
 &  & PARTIAL / ALL / BLIND &  & 21.82\% \\ \cline{3-3} \cline{5-5} 
 &  & PARTIAL / BLIND / ALL &  & 3.03\% \\ \hline
\end{tabular}
\end{table}

\begin{figure}
  \includegraphics[width=\linewidth]{./images/drift_reset_type_rankings_pie}
\caption{\label{fig:drift_reset_type_rankings_pie}Pie chart illustrating table \ref{table:drift_reset_type_rankings}}
\end{figure}

\subsubsection{Ground Truth}

Let us now investigate the Ground Truth parameter (\textbf{which indicates the percentage of labelled instances used to train our ensemble}). Table \ref{table:nemenyi_significant_breakdown} shows that less than one third of parameter combinations show a statistical significance difference for the $\kappa_t$ metric; but over $84\%$ for the execution time metric. This suggests that the value of the ground truth parameter has a measurable impact on the execution time, across a large portion of the other parameters, (almost) no matter their values.

Table \ref{table:ground_truth_rankings} \textbf{shows that}, for the $\kappa_t$ metric, there are multiple possible options for ranking the parameter values. For those with a statistical significant difference, 100 performs better than 60. 100 ranks first across all parameter combinations over $98\%$ of the time, as expected. This is normal, as we expect our ensemble to better learn to model the streamed data from ground truth rather than from its own predictions of class label values, considering that it can enable our ensemble to learn from erroneously labelled data. It is a good sign, however, that there is no statistical significant difference for parameter values other than 100 and 60. It is logical that selecting a higher percentage of ground truth will lead to higher predictive accuracy, but we can allow for a trade-off between predictive accuracy and time.

\textbf{We can also see}, from table \ref{table:ground_truth_rankings}, that there are a lot of different combinations for ranks and statistical significant different values for the execution time metric. $60\%$ ground truth ranks first across $74.31\%$ of parameter combinations, and $100\%$ ground truth ranks first across only $17.71\%$ of parameter values. When we exclude insignificant results, our previous observation remains true, but across a smaller percentage of parameter combinations, but only by $5\%$ to $10\%$. We can conclude that the using $60\%$ ground truth is more likely to decrease execution time. This is an unexpected finding, as we expected that using less ground truth would cause an increase in execution time by causing more drift detection events and therefore more model resets and retrains. However, it is possible that by using less ground truth, there are no drift detection events because the model is not able to properly learn from the data, so the model is never reset or retrained which might also explain why this parameter value also ranks as one of the worst in predictive accuracy.

When looking at the raw values, as \textbf{depicted} in figure \ref{fig:order_by_ground_truth}, we notice that the $\kappa_t$ metric does not change much over the parameter combinations, but rises noticeably starting when using $80\%$ ground truth. The regular peaks and valleys within each set of ground truth values (separated by black vertical lines) simply represent how other parameters affect $\kappa_t$. We also notice that the amount of ground truth has a strong effect over the sine1, mixed and particularly on the circles data sets. It may be that it is easier to model the data from SEA than from the other data sets, and therefore it does not need as much ground truth to build an accurate model. Not much can be said when looking at the raw values for the execution time metric other than that there may be fewer valleys in the section dedicated for the $60\%$ ground truth parameter.

\begin{table}[]
\centering
\caption{\label{table:ground_truth_rankings}Rankings for ground truth and parameter combination counts}
\begin{tabular}{|l|c|l|c|c|}
\hline
\textbf{Metric} & \textbf{Stat. Sig.} & \textbf{Ranks} & \textbf{Stat. Sig. values} & \textbf{\%} \\ \hline
\multirow{4}{*}{$\kappa_t$} & \checkmark & 100 / x / x / x / 60 & 100 / 60 & 28.80\% \\ \cline{2-5} 
 & \multirow{3}{*}{$\times$} & 100 / x / x / x / 60 & \multirow{3}{*}{} & 65.15\% \\ \cline{3-3} \cline{5-5} 
 &  & 100 / x / x / 60 / 70 &  & 4.55\% \\ \cline{3-3} \cline{5-5} 
 &  & 80 / 100 / 90 / 70 / 60 &  & 1.52\% \\ \hline
\multirow{21}{*}{Execution time} & \multirow{12}{*}{\checkmark} & 100 / 60 / x / x / 70 & 100 / 70 & 9.10\% \\ \cline{3-5} 
 &  & 100 / 60 / 90 / 80 / 70 & \begin{tabular}[c]{@{}c@{}}100 / 70\\ 60 / 70\end{tabular} & 0.51\% \\ \cline{3-5} 
 &  & 100 / 60 / 90 / 70 / 80 & 100 / 80 & 2.02\% \\ \cline{3-5} 
 &  & 100 / 60 / 80 / 70 / 90 & 100 / 90 & 0.51\% \\ \cline{3-5} 
 &  & 60 / x / x / x / 100 & 60 / 100 & 14.16\% \\ \cline{3-5} 
 &  & 60 / x / x / x / 70 & 60 / 70 & 1.02\% \\ \cline{3-5} 
 &  & 60 / x / x / x / 80 & 60 / 80 & 50.51\% \\ \cline{3-5} 
 &  & 60 / 70 / 100 / 80 / 90 & \begin{tabular}[c]{@{}c@{}}60 / 80\\ 60 / 90\end{tabular} & 0.51\% \\ \cline{3-5} 
 &  & 60 / 80 / 70 / 100 / 90 & 60 / 90 & 1.01\% \\ \cline{3-5} 
 &  & 70 / 90 / 80 / 100 / 60 & \begin{tabular}[c]{@{}c@{}}70 / 60\\ 90 / 60\end{tabular} & 0.51\% \\ \cline{3-5} 
 &  & 90 / x / x / x / 100 & 90 / 100 & 2.03\% \\ \cline{3-5} 
 &  & 90 / 70 / 80 / 100 / 60 & 90 / 60 & 3.03\% \\ \cline{2-5} 
 & \multirow{9}{*}{$\times$} & 100 / 60 / x / x / 90 & \multirow{9}{*}{} & 1.02\% \\ \cline{3-3} \cline{5-5} 
 &  & 100 / 60 / x / x / 70 &  & 1.52\% \\ \cline{3-3} \cline{5-5} 
 &  & 100 / 60 / 90 / 70 / 80 &  & 3.03\% \\ \cline{3-3} \cline{5-5} 
 &  & 60 / x / x / 80 / 90 &  & 1.52\% \\ \cline{3-3} \cline{5-5} 
 &  & 60 / x / x / x / 80 &  & 3.55\% \\ \cline{3-3} \cline{5-5} 
 &  & 60 / 100 / 90 / 80 / 70 &  & 0.51\% \\ \cline{3-3} \cline{5-5} 
 &  & 60 / x / x / 80 / 100 &  & 1.52\% \\ \cline{3-3} \cline{5-5} 
 &  & 90 / x / x / x / 100 &  & 2.03\% \\ \cline{3-3} \cline{5-5} 
 &  & 90 / 70 / 80 / 100 / 60 &  & 0.51\% \\ \hline
\end{tabular}
\end{table}

\begin{figure}
  \includegraphics[width=\linewidth]{./images/ground_truth}
\caption{\label{fig:order_by_ground_truth}$\kappa_t$ across all parameter combinations, ordered by ground truth}
\end{figure}

\subsubsection{Voting Type}

And finally, we will investigate the Voting Type parameter. As we can see from table \ref{table:nemenyi_significant_breakdown}, none of the 180 parameter combinations proved to show a significant statistical difference for the $\kappa_t$ metric. However, a large portion of the parameter combinations showed a significant statistical significance for the execution time parameter ($85.56\%$ of combinations). This leads us to believe that the voting type does not have a major influence on the prediction accuracy. However, for execution time, there is a statistical significant difference present across a good majority of parameter combinations, leading us to believe that the significance remains true across a good portion of other parameter values.

As we can see from table \ref{table:nemenyi_significant_breakdown_aggregate}, there are not very many different possible rankings for both metrics. Indeed, there are 4 possible rankings for $\kappa_t$ and 5 for the execution time metric.

Let us now examine table \ref{table:voting_type_rankings}, or better yet the pie charts from figure \ref{fig:voting_type_rankings_pie} for the \textit{Voting Type} parameter.
We will first examine the results for the execution time. At first glance, it is clear that \textit{probability voting} ranks first amongst the majority of parameter combinations ($90\%$ of them), while \textit{weighted averaged probability voting} ranks first amongst the remaining $10\%$ of parameter combinations. We can say with the utmost certainty that \textit{averaged weighted probability voting} is not a good choice for a parameter value if the goal is to keep execution time low, as it ranks last over $99\%$ of parameter combinations. It makes sense that probability voting ranks first in execution time over most parameter combinations as its implementation is such that the two other voting schemes add to the operations computed for probability voting. In other words, for the other two voting schemes, the results obtained through probability voting is an intermediate result.

For the $\kappa_t$ metric, \textit{probability voting} ranks first across about $69\%$ of parameter combinations, and \textit{weighted averaged probability voting} ranking first across the remaining fraction of parameter combinations. Probability voting appears to rank better than the other two voting schemes, but this could be due to the noise that we add when we apply the weights to the prediction probabilities. Furthermore, the ranks do not allow us to determine by how much the voting scheme impacts the actual metrics. Then, to answer why \textit{weighted averaged probability voting} performs better than \textit{averaged weighted probability voting}, it could be that there is less noise introduced by first averaging the prediction probabilities then applying the weight, than doing those operations in the opposite way.

When we look at the raw results from our simulations, and order them by voting type (then by ground truth), as seen in figure \ref{fig:order_by_voting_type}, we can notice that there is no significant difference in the $\kappa_t$ values between probability voting and weighted averaged probability voting. However, we can see a significant difference between probability voting and averaged weighted probability voting.

The best choice here appears to be \textit{probability voting}, but we will be able to better determine the veracity of this statement when ranking all parameter combinations and comparing their raw metric values in the next section. It does not matter whether execution time or predictive accuracy matters more, probability voting is more likely to better model the data, predict new instances all the while taking the least amount of time to do so, as opposed to the other parameter values.

\begin{table}[]
\centering
\caption{\label{table:voting_type_rankings}Rankings for voting type and parameter combination counts}
\begin{tabular}{|l|c|l|c|c|}
\hline
\textbf{Metric} & \textbf{Stat. Sig.} & \textbf{Ranks} & \textbf{Stat. Sig. values} & \textbf{\%} \\ \hline \hhline{=====}
\multirow{4}{*}{$\kappa_t$} & \multirow{4}{*}{$\times$} & PROBA / AVG W / W AVG & \multirow{4}{*}{} & 1.11\% \\ \cline{3-3} \cline{5-5} 
 &  & PROBA / W AVG / AVG W &  & 67.78\% \\ \cline{3-3} \cline{5-5} 
 &  & W AVG / AVG W / PROBA &  & 1.11\% \\ \cline{3-3} \cline{5-5} 
 &  & W AVG / PROBA / AVG W &  & 30.00\% \\ \hline \hhline{=====}
\multirow{5}{2cm}{Execution time} & \multirow{2}{*}{\checkmark} & PROBA / W AVG / AVG W & PROBA / AVG W & 78.33\% \\ \cline{3-5} 
 &  & W AVG / PROBA / AVG W & W AVG / AVG W & 7.22\% \\ \cline{2-5} 
 & \multirow{3}{*}{$\times$} & PROBA / AVG W / W AVG & \multirow{3}{*}{} & 0.56\% \\ \cline{3-3} \cline{5-5} 
 &  & PROBA / W AVG / AVG W &  & 11.11\% \\ \cline{3-3} \cline{5-5} 
 &  & W AVG / PROBA / AVG W &  & 2.78\% \\ \hline
\end{tabular}
\end{table}

\begin{figure}
  \includegraphics[width=\linewidth]{./images/voting_type_rankings_pie}
\caption{\label{fig:voting_type_rankings_pie}Pie chart \textbf{illustrating} table \ref{table:voting_type_rankings}}
\end{figure}

\begin{figure}
  \includegraphics[width=\linewidth]{./images/raw_voting_type}
\caption{\label{fig:order_by_voting_type}$\kappa_t$ across all parameter combinations, ordered by voting type}
\end{figure}

\subsection{Summary}
The above results suggest that it is preferable to use the following parameter combination to obtain higher $\kappa_t$ values: [\textit{sliding, probability, 25, $100\%$, all, 1e}].
\textbf{Otherwise,} to minimise the execution time, the results suggest to use [\textit{hybrid, probability, 100, $60\%$, all, 1e / 1c}].

The differences lie with the window type, the batch size, the ground truth used and partially the drift detector count. The results above indicated that using probability voting,  1 detector per ensemble to detect drifts, and resetting all classifiers when drifts occur would lead to better $\kappa_t$ values and a lower execution time. For the batch size, window type, and the drift detector count, it is completely logical that choosing one value over another would change the execution time as they were, at least partially, implemented as time saving measures.

In the following section, we will rank the parameter combinations to determine if the ones listed two paragraphs above are truly top ranking.

\section{Comparing all parameter combinations}
In this section, we aim to rank all 1110 parameter combinations \textbf{to determine which parameters perform the best, and those that perform the worst}. This will be done in three different configurations:
\begin{enumerate}
\item over $\kappa_t$
\item over execution time
\item over both metrics
\end{enumerate}

For figures \ref{fig:rank_kappa} and \ref{fig:rank_seconds}, the parameter values are shortened to improve readability.
The values are split by vertical lines. See table \ref{table:ranking_parameter_values_mapping} for the mapping between shortened and real parameter values. The order of the parameters in the figures are the following: [\textit{window type, voting type, ground truth, batch size, drift reset type, detector count, detector content}].

\begin{table}[]
\centering
\caption{\label{table:ranking_parameter_values_mapping}Mapping shortened parameter values with full name}
\begin{tabular}{|l|c|l|}
\hline
\textbf{Parameter} & \textbf{Value} & \textbf{Real Value} \\ \hline
\multirow{2}{*}{Window type} & s & Sliding \\ \cline{2-3} 
 & h & Hybrid \\ \hline
\multirow{4}{*}{Voting type} & aw & Averaged weighted probability \\ \cline{2-3} 
 & b & Boolean \\ \cline{2-3} 
 & p & Probability \\ \cline{2-3} 
 & wa & Weighted averaged probability \\ \hline
\multirow{4}{*}{Drift reset type} & a & All \\ \cline{2-3} 
 & b & Blind \\ \cline{2-3} 
 & n & None \\ \cline{2-3} 
 & p & Partial \\ \hline
\multirow{2}{*}{Detector count} & 1c & 1 per classifier \\ \cline{2-3} 
 & 1e & 1 for ensemble \\ \hline
\multirow{3}{*}{Detector content} & b & Boolean \\ \cline{2-3} 
 & p & Probability \\ \cline{2-3} 
 & wp & Weighted probability \\ \hline
\end{tabular}
\end{table}

\subsection{Ranking over $\kappa_t$}
Figure \ref{fig:rank_kappa} shows the post-hoc Nemenyi graph ranking all parameter combinations. Due to the sheer number of parameter combinations, we cannot use the rankings to specify whether a pair of parameter combinations show a significant statistical difference. It does, however, allow us to see which are the top ranking parameter combinations for our algorithm. On the left hand side are the best ranking parameter combinations for the $\kappa_t$ metric, and on the right hand side are the worst performing ones.
We'll consider the 50 top ranking parameter combinations, so the figure was cropped to reduce the size and aid readability.

\begin{figure}
  \includegraphics[width=\linewidth]{./images/rank_kappa_cropped}
\caption{\label{fig:rank_kappa}Ranking all parameter combinations, over $\kappa_t$}
\end{figure}
\begin{table}[]
\centering
\caption{\label{table:rank_kappa_breakdown}Breakdown of parameter value frequency in the top 50 ranked parameter combinations for $\kappa_t$}
\begin{tabular}{|l|c|c|}
\hline
\textbf{Parameter} & \textbf{Value} & \textbf{Breakdown} \\ \hline \hhline{===}
\multirow{2}{*}{Sliding type} & h & 10\% \\ \cline{2-3} 
 & s & 90\% \\ \hline
\multirow{3}{*}{Voting type} & aw & 36\% \\ \cline{2-3} 
 & p & 26\% \\ \cline{2-3} 
 & wa & 38\% \\ \hline
\multirow{2}{*}{Ground truth} & 90 & 38\% \\ \cline{2-3} 
 & 100 & 62\% \\ \hline
\multirow{3}{*}{Batch size} & 25 & 34\% \\ \cline{2-3} 
 & 75 & 38\% \\ \cline{2-3} 
 & 100 & 28\% \\ \hline
\multirow{4}{*}{Drift reset type} & a & 26\% \\ \cline{2-3} 
 & b & 26\% \\ \cline{2-3} 
 & n & 12\% \\ \cline{2-3} 
 & p & 36\% \\ \hline
\multirow{3}{*}{Detector count} & 1c & 42\% \\ \cline{2-3} 
 & 1e & 46\% \\ \cline{2-3} 
 & none & 12\% \\ \hline
\multirow{3}{*}{Detector content} & p & 68\% \\ \cline{2-3} 
 & wp & 20\% \\ \cline{2-3} 
 & none & 12\% \\ \hline
\end{tabular}
\end{table}

Table \ref{table:rank_kappa_breakdown} indicates how frequently each parameter occurred in the top 50 ranked parameter combinations for $\kappa_t$. These results suggest that parameter combinations with the following values to score higher on the $\kappa_t$ metric: [\textit{sliding, *, $100\%$, 75 or 25, partial, 1c or 1e, probability}]. The asterisk is used to indicate any value for that parameter.
These findings don't exactly match up with the results found in the previous section. For example, we found that probability voting was more likely to be top ranking in the previous section, whereas we found in this section that weighted averaged or averaged weighted probability voting was more frequently in the 50 top ranked parameter combinations. The same can be said for the drift reset type parameter value.

\subsection{Ranking over execution time}
Figure \ref{fig:rank_seconds} shows the post-hoc Nemenyi ranking results for the execution time metric. As before, the left hand side lists the best ranking parameter combinations, and the right hand side shows the worst performing ones. 
We'll consider the 50 top ranking parameter combinations.

\begin{figure}
  \includegraphics[width=\linewidth]{./images/rank_seconds_cropped}
\caption{\label{fig:rank_seconds}Ranking all parameter combinations, over execution time}
\end{figure}
\begin{table}[]
\centering
\caption{\label{table:rank_seconds_breakdown}Breakdown of parameter value frequency in the top 50 ranked parameter combinations for execution time}
\begin{tabular}{|l|c|c|}
\hline
\textbf{Parameter} & \textbf{Value} & \textbf{Breakdown} \\ \hline \hhline{===}
\multirow{2}{*}{Sliding type} & h & 78\% \\ \cline{2-3} 
 & s & 22\% \\ \hline
\multirow{4}{*}{Voting type} & aw & 0\% \\ \cline{2-3} 
 & b & 4\% \\ \cline{2-3} 
 & p & 56\% \\ \cline{2-3} 
 & wa & 40\% \\ \hline
\multirow{5}{*}{Ground truth} & 60 & 80\% \\ \cline{2-3} 
 & 70 & 4\% \\ \cline{2-3} 
 & 80 & 4\% \\ \cline{2-3} 
 & 90 & 4\% \\ \cline{2-3} 
 & 100 & 8\% \\ \hline
\multirow{3}{*}{Batch size} & 25 & 0\% \\ \cline{2-3} 
 & 75 & 42\% \\ \cline{2-3} 
 & 100 & 58\% \\ \hline
\multirow{4}{*}{Drift reset type} & a & 42\% \\ \cline{2-3} 
 & b & 14\% \\ \cline{2-3} 
 & p & 36\% \\ \cline{2-3} 
 & none & 8\% \\ \hline
\multirow{3}{*}{Detector count} & 1c & 28\% \\ \cline{2-3} 
 & 1e & 64\% \\ \cline{2-3} 
 & none & 8\% \\ \hline
\multicolumn{1}{|c|}{\multirow{4}{*}{Detector content}} & b & 4\% \\ \cline{2-3} 
\multicolumn{1}{|c|}{} & p & 74\% \\ \cline{2-3} 
\multicolumn{1}{|c|}{} & wp & 14\% \\ \cline{2-3} 
\multicolumn{1}{|c|}{} & none & 8\% \\ \hline
\end{tabular}
\end{table}

Table \ref{table:rank_seconds_breakdown} indicates how frequently each parameter occurred in the top 50 ranked parameter combinations for the execution time. These results suggest that parameter combinations with the following values to have lower values for the execution time metric: [\textit{hybrid, probability, $60\%$, 100, partial, 1c or 1e, probability}]. These results almost completely match the results from the previous section, confirming our findings.

\subsection{Ranking over both metrics}

By filtering the raw values from the simulations we ran, in combination with the rankings obtained in the two previous subsections, the best performing parameter combinations will be determined in this subsection.

Figure \ref{fig:rank_both_all} shows the raw values for each data set for both measured metrics. The values are ordered by an average of the ranks for $\kappa_t$ and execution time. To give more weight to the predictive accuracy in the ordering, the equation for the average is the following: $\frac{3\times rank_{\kappa_t}+rank_{execution\ time}}{4}$.

\begin{figure}
  \includegraphics[width=\linewidth]{./images/rank_both}
\caption{\label{fig:rank_both_all}Raw $\kappa_t$ and execution time values ordered by averaged ranks}
\end{figure}

In order to find the right balance, we started to filter out the parameter combinations with the conditions found in table \ref{table:rank_both_filter_dataset}, and then filtered out those whose averaged rank was below 340. We should note that the $\kappa_t$ ranks are in the range of $[228.8, 947.3]$ and that of the execution time in the range of $[1, 1108.2]$. Figure \ref{fig:compare_both_best} shows the remaining parameter combinations, with their raw values for both measured metrics. These filters were selected by intuition and through extensive exploration and inspection, in order to remove very high execution times as well as poor $\kappa_t$ values. We also chose to include the parameter combinations that led to the best overall $\kappa_t$ average metric, and the one with the best average execution time metric.

\begin{table}[]
\centering
\caption{\label{table:rank_both_filter_dataset}Data set filtering conditions}
\begin{tabular}{|l|c|} 
\hline
\textbf{Data set} & \textbf{Condition} \\ \hline \hhline{==}
SEA $0\%$ noise & $\le$ 11 seconds \\ \hline
circles & $\le$ 9.2 seconds \\ \hline
sine & $\le$ 9.2 seconds \\ \hline
mixed & $\le$ 9.1 seconds \\ \hline
\end{tabular}
\end{table}

\begin{figure}
  \includegraphics[width=\linewidth]{./images/compare_both_best}
\caption{\label{fig:compare_both_best}Remaining parameter combinations and their raw metric values after filtering}
\end{figure}

For the best resulting predictive accuracy, [\textit{sliding window, probability voting, $100\%$ ground truth, 100 batch size, partial reset, 1 drift detector, probability drift content}] proved to be the best parameter combination. And for the best execution time,  the following combination of parameters proved to be the best: [\textit{hybrid window, probability, $60\%$ ground truth, 100 batch size, blind reset, 1 drift detector, probability content}]. 

However, if we can accept a $1$ to $4\%$ reduction for $\kappa_t$ values, then significant time savings can be achieved by using [\textit{hybrid window, probability voting, $100\%$ ground truth, 75 batch size, reset all, 1 drift detector, probability content}]. Indeed, we can reduce execution time by up to $9.5\%$.

\subsection{Effects of reducing ground truth used for training}
As stated in section \ref{section:vc_reduce_gt}, we want to determine at what ratio of predictions to ground-truth our voting ensemble's prediction accuracy declines and by how much. Figure \ref{fig:ground_truth_drop} shows examples of parameter combinations using varying amounts of ground truth ($10\%$ increments starting at 60). The values shown in the graphs were selected manually after having been ranked with the averaging equation shown above. The selection process was simple, we selected one or two parameter combinations that had the highest averaged rank value for a given percentage of ground truth. The graph indicates that the predictive accuracy of our ensemble doesn't drop until we use $80\%$ of ground truth. The execution times can be reduced by further reducing the predictive accuracy.

One finding that we find particularly odd is that as the use of ground truth diminishes, the predictive accuracy increases for the SEA generated data sets with noise. We are led to believe that for increasing levels of noise in a data set, reducing the ground truth used (to an extent) to train a model increases its predictive accuracy. Further research, outside the scope of this thesis, is needed to ascertain the veracity of this finding.

\begin{figure}
  \includegraphics[width=\linewidth]{./images/chapter5/ground_truth}
\caption{\label{fig:ground_truth_drop}$\kappa_t$ and execution times for parameter combinations using varying amounts of ground truth}
\end{figure}

\section{Comparing to the State of the Art}

Now that we have determined which parameter combinations worked particularly well, we can compare them to the State of the Art.

As previously mentioned, the algorithms which we will be comparing our voting ensemble to will be mainly the Leveraging Bagging algorithm. As we explained in the previous chapter, the Leveraging Bagging will be comprised of 10 Hoeffding Tree estimators, each with its own ADWIN drift detector. We will also be using a regular Hoeffding Tree using the default parameters without any drift detector. Finally, we will also include a no-change and a majority class classifier in our comparison.

\subsection{Choosing a window size for State of the Art algorithms}

\begin{figure}
  \includegraphics[width=\linewidth]{./images/chapter5/compare_sota}
\caption{\label{fig:raw_compare_sota}$\kappa_t$ and execution times of State of the Art algorithm with varying window sizes}
\end{figure}

For these algorithms, we made sure to use $100\%$ ground truth for the training, sliding windowing and only modified the window size. However, changing the window size did not change the execution times or $\kappa_t$ by much more than $1\%$ or 1 second as can be seen in figure \ref{fig:raw_compare_sota}. For this reason, we chose to simply keep one example for each algorithm, that ranked better with a given window size than other. We should note that applying the Friedman test and Nemenyi tests showed that the window size resulted in confirming the null hypothesis that all window sizes led to similar results for each classifier for the no change, majority voting, and SGD classifiers. Therefore, we simply took the best overall ranking window size. For Hoeffding Trees, window size of 25 showed a significant statistical difference to 100 but only for the execution time. For Leveraging Bagging, window size of 25 showed a significant statistical difference to 100 but only for the $\kappa_t$ metric.

The resulting chosen window sizes are as follows: no change (25),  majority voting (25), SGD (75), Hoeffding Tree (25), Leveraging Bagging (25).

We have opted to compare these algorithms to our voting ensemble with 6 different parameter combinations (1 for each increment of ground truth used, and an additional one using $100\%$ ground truth).


\subsection{Visual comparison}

Finally, we can compare our voting ensemble to the state of the art. Figure \ref{fig:raw_compare_sota_all} shows the raw results, to better visualize how each algorithm, and its parameter combinations affects the data sets that they are trying to model.

\begin{figure}
  \includegraphics[width=\linewidth]{./images/chapter5/compare_sota_all}
\caption{\label{fig:raw_compare_sota_all}$\kappa_t$ and execution times when comparing our Voting Ensemble to the State of the Art}
\end{figure}

As we can see from these two graphs, Leveraging Bagging (LB) does achieve the best predictive accuracy but the worst execution time. While the difference in predictive accuracy between LB and the other algorithms is noticeable, it isn't glaring. However, when it comes to execution time, we were required to use a logarithmic scale to show its run time while also showing the run times of other algorithms. LB takes more than 2 orders of magnitude longer than the Voting Ensemble, and 1.5 order of magnitude longer than a Hoeffding Tree (HT). Given that LB is comprised of 10 HTs, it makes perfect sense that LB takes so much longer to run.

However, our findings from the graph do not have the weight of a proper statistical analysis, which follows in the next section.

\subsection{Statistical Analysis}
In this final section, we will test the following two null hypotheses:
\begin{enumerate}
\item all algorithms, with their respective parameters predict classes equally well ($\kappa_t$)
\item all algorithms, with their respective parameters run in an equal amount of time.
\end{enumerate}

\subsubsection{For $\kappa_t$}

We will start with the first, using the $\kappa_t$ metric. Again, the Friedman test was used, with a significance level of 0.05. We found that $p < 2.1\times10^{-23}$, thus rejecting the null hypothesis.
To determine which pairs of algorithms actually differ, we used the post-hoc Nemenyi test, yet again. The results can be seen in figure \ref{fig:sota_compare_all_kappa_nemenyi}, where a lower rank means a better predictive accuracy (a better $\kappa_t$).

\begin{figure}
  \includegraphics[width=\linewidth]{./images/chapter5/sota_compare_all_kappa_nemenyi}
\caption{\label{fig:sota_compare_all_kappa_nemenyi}Nemenyi graph ranking $\kappa_t$ for various algorithms}
\end{figure}

A Nemenyi graph shows a ranking of algorithms on a scale from 1 to N (typically the number of algorithms compared). A bar labelled critical difference (CD) is shown above the scale, which is the minimum rank length for two algorithms to not show a significant statistical difference in rank. 
Additionally, there may be horizontal bars that link ranked algorithms. Any algorithms that share a same horizontal bar are not significantly statistically different. Pairs of algorithms that are further apart than the CD bar are significantly statistically different.

We can see from the graph that there is not significant statistical difference between LB, our Voting Ensemble using our best overall parameter combination, an SGD classifier, our Voting Ensemble using our hybrid windowing approach, and our Voting Ensemble using only $80\%$ ground truth when training. This confirms the rejection of the null hypothesis for $\kappa_t$. It's also a very good sign, because all of the above mentioned algorithms were statistically significantly better than a single Hoeffding Tree.

As a side note, we can also say that there is a significant statistical difference between both the majority voting and no change classifiers with all algorithms, aside from our Voting Ensemble training with $70\%$ of ground truth or less.

Therefore, this test showed that our Voting Ensemble, using our preferred parameter combinations, did not perform better or worse, statistically speaking, than Leveraging Bagging.

\subsubsection{For execution time}

For the final measure, execution time, we use the Friedman test, with a significance level of 0.05. We found that $p < 1.68\times10^{-31}$, thus leaving no doubt as to the rejection of the null hypothesis. The post-hoc Nemenyi test is used to determine which pairs of algorithms differ. The Nemenyi graph is shown in figure  \ref{fig:sota_compare_all_execution_time_nemenyi}, where a lower rank means a lower execution time.

\begin{figure}
  \includegraphics[width=\linewidth]{./images/chapter5/sota_compare_all_execution_time_nemenyi}
\caption{\label{fig:sota_compare_all_execution_time_nemenyi}Nemenyi graph ranking execution times for various algorithms}
\end{figure}

We can see from the graph that Leveraging Bagging ranks last, and Hoeffding Trees ranks second last, which is as expected given the raw values that we saw above. 
The graph also shows that there is a significant statistical difference between Leveraging Bagging (the State of the Art algorithm we're comparing), and our Voting Ensemble (except when using $70\%$ or $80\%$ ground truth). Given that Leveraging Bagging runs in over 2 orders of magnitude longer than our Voting Ensemble, this result is not surprising in the least. It is, however, comforting to have our visual analysis backed by this statistical test.

Therefore, this test statistically showed that our Voting Ensemble runs significantly faster than Leveraging Bagging.

Additional graphs (statistical significance heatmaps and more Nemenyi graphs) can be seen in the appendix: \ref{section:nemenyi-graphs-statistical-analysis}.

\subsubsection{What do these results mean ?}
First of all, our statistical significance tests showed that our Voting Ensemble was able to outperform the State of the Art \textit{Leveraging Bagging} algorithm in execution time, and that it was able to perform on par with \textit{Leveraging Bagging} in regards to the $\kappa_t$ measured metric. It also showed that we could use only $90\%$ ground truth without compromising our Ensemble's predictive accuracy in comparison to \textit{Leveraging Bagging}.

Our algorithm therefore predicts on par with Leveraging Bagging and brings outstanding time savings in algorithm run-time, running somewhere over 160 times faster.

Practically, this means that ensembles should definitely be considered when execution time is an important metric. However, for the applications that strictly require high predictive accuracy, Leveraging Bagging would still be the valid choice.

\section{General discussion and conclusion}

% Chapter 6

\chapter{Conclusion\label{chapter:conclusion}} % Main chapter title

%----------------------------------------------------------------------------------------

This thesis focused on improving semi-supervised learning from evolving streams without the use of clustering techniques, which are computationally expensive \cite{krempl2014open}. The goal of this study was to design fast algorithms to work with fewer labelled examples, and to extend an existing algorithm to detect drifts without relying on ground truth. Experiments were conducted in order to compare the performance of our framework against that of the state of the art in terms of predictive accuracy and execution time, while considering the percentage of labelled instances used at each stage of learning. This chapter discusses our contributions and presents opportunities for future work.

%----------------------------------------------------------------------------------------
 
\section{Contributions}
A great deal of research is being conducted to develop algorithms for supervised learning of evolving streams. These techniques are usually ensemble-based, and typically make use of either boosting or bagging. This research is necessary to understand how well an algorithm can learn from an evolving stream, but the use of supervised training is to online learning as training wheels are to learning how to ride a bicycle. In the real-world, the assumption that true labels will arrive on time is unequivocally unreasonable. We introduce our semi-supervised hybrid-windowing ensembles for learning in evolving streams as our solution to deal with this issue without using clustering techniques.

We first proposed a voting ensemble that uses a modified soft voting approach, by weighting each classifier’s predictive confidence. Next, we developed a novel windowing type for ensembles, as sliding windows are very time consuming and regular tumbling windows are not a suitable replacement. Our windowing technique can be considered a hybrid of the two: we train each sub-classifier in the ensemble with tumbling windows, but delay training in such a way that only one sub-classifier can update its model per iteration. We also extended selective self-training by ignoring its heuristic: training classifiers using all predicted examples, and not only those with high predictive confidence. Finally, we extended an existing concept drift detector to successfully operate without any labelled data, by using a sliding window of our ensemble’s prediction confidence, instead of a boolean value indicating whether, or not, the ensemble predicted correctly.

%----------------------------------------------------------------------------------------

\section{Future Work}
Our framework took very little execution time, far below that of a current state of the art technique, and achieved comparable predictive accuracy to that of state of the art techniques that trained on fully-labelled data sets, while ours only trained on a subset of those labels, and ignoring them completely for detecting drifts. We believe that a higher predictive accuracy can be achieved without significantly impacting the execution time, even when training with less labelled data.

In order to reduce the quantity of labelled data used for (pre-)training, we can introduce specialised domain knowledge into the learning cycle. An interactive active learning approach can be used, as seen in \cite{floyd2017activetext}, where an intuitive online interface is used to request labels, from an oracle, for any number of uncertain data instances to train on instances that are representative of the stream, or those that classifiers find difficult.

Furthermore, our framework can be improved by incorporating ideas that have been proven to work such as intelligent window sizes, and replacing classifiers with a bad predictive accuracy streak. To deal with recurring concepts, we can incorporate weighted summarising classifiers as seen in Learn$^{++}$.NSE \cite{elwell2011incremental}, however, the memory issue would need to be addressed.

Additionally, our framework does not guarantee diversity among the classifiers in the ensemble. A possible approach would be to replace the cyclic training from hybrid windows with a stochastic method. In other words, instead of training the classifiers in the ensemble in the same cyclical sequential manner, a classifier would be chosen at random to be trained from the new hybrid window.

Another area to be explored is how our framework performs when dealing with mixed concept drifts, and perform a study using a real-world data set.

Finally, our framework can be extended to better deal with unbalanced datasets. This would increase its applicability to real-world problems such as intrusion detection, and fraud detection.

%----------------------------------------------------------------------------------------
%	THESIS CONTENT - APPENDICES
%----------------------------------------------------------------------------------------

\appendix % Cue to tell LaTeX that the following "chapters" are Appendices

% Include the appendices of the thesis as separate files from the Appendices folder
% Uncomment the lines as you write the Appendices

% \include{Appendices/AppendixA}
%\include{Appendices/AppendixB}
%\include{Appendices/AppendixC}
\chapter{Graphs} % Main chapter title

\label{Appendix}

%----------------------------------------------------------------------------------------

\section{\label{section:nemenyi-graphs-statistical-analysis}Nemenyi Graphs}

\begin{figure}
  \includegraphics[width=\linewidth]{./images/appendix/heatmap_nemenyi_graphs/kappa_t_heatmap}
\caption{\label{fig:sota_kappa_t_heatmap}State of the Art comparison: $\kappa_t$ heatmap}
\end{figure}

\begin{figure}
  \includegraphics[width=\linewidth]{./images/appendix/heatmap_nemenyi_graphs/sota_compare_all_kappa_nemenyi__0_99}
\caption{\label{fig:sota_kappa_t_099}State of the Art comparison: post-hoc Nemenyi graph for $\kappa_t$, $\alpha=0.01$}
\end{figure}

\begin{figure}
  \includegraphics[width=\linewidth]{./images/appendix/heatmap_nemenyi_graphs/sota_compare_all_kappa_nemenyi__0_999}
\caption{\label{fig:sota_kappa_t_0999}State of the Art comparison: post-hoc Nemenyi graph for $\kappa_t$, $\alpha=0.001$}
\end{figure}

\begin{figure}
  \includegraphics[width=\linewidth]{./images/appendix/heatmap_nemenyi_graphs/seconds_heatmap}
\caption{\label{fig:sota_seconds_heatmap}State of the Art comparison: execution time heatmap}
\end{figure}

\begin{figure}
  \includegraphics[width=\linewidth]{./images/appendix/heatmap_nemenyi_graphs/sota_compare_all_execution_time_nemenyi__0_99}
\caption{\label{fig:sota_seconds_099}State of the Art comparison: post-hoc Nemenyi graph for execution time, $\alpha=0.01$}
\end{figure}

\begin{figure}
  \includegraphics[width=\linewidth]{./images/appendix/heatmap_nemenyi_graphs/sota_compare_all_execution_time_nemenyi__0_99}
\caption{\label{fig:sota_seconds_0999}State of the Art comparison: post-hoc Nemenyi graph for execution time, $\alpha=0.001$}
\end{figure}

\chapter{Summaries}
As recommended by Silyn in \cite[p.~151]{silyn2012writing}, we have created this chapter to collect all of the chapter summaries.

%\section{Chapter \ref{chapter:introduction}: Introduction}

\section{Chapter \ref{chapter:background_work}: Background Work}
This chapter presents the fundamentals of machine learning and the various challenges of extracting knowledge from data streams. We introduce classification as a type of machine learning, for which the goal is to extract knowledge, computationally, from labelled data. Algorithms are developed to model the relationship between the data and the class labels. Semi-supervised learning is also a classification task but with the added constraint of learning from a data set that is not entirely labelled. In this thesis, the data we learn from come from data streams, which are voluminous, volatile and velocious. As such, constraints on time and memory usage must be respected, and a mechanism is required to forget "old data" safely.

We cover specific classifiers: baseline classifiers are usually relatively simple and used as a baseline for comparing classifier performance. We also cover the algorithms that we employed in our framework, as well as the state of the art: Naive Bayes, Stochastic Gradient Descent, Hoeffding Trees and Leveraging Bagging.

We define concept drifts as an evolution in the probability distribution of classes and/or attributes. We describe the main types of concept drifts that can occur, regardless of how self-explanatory they are named: abrupt, gradual and recurring. We then review drift detection strategies presented in the literature, including FHDDM/S which we extend in this thesis. Our review shows a gap in research as it pertains to semi-supervised drift detection.

Next, we define ensembles as an amalgamation of any number of classifiers. As all classifiers must output a prediction, ensembles must as well; we present existing techniques to map these multiple outputs to a single one. Having defined ensembles, we review those that were proposed in the literature using two criteria: the processing method and whether or not they were designed to deal with evolving concepts. The processing method distinguishes if instances are analyzed online (one-by-one) or in batches. Our review shows a gap in research as it pertains to semi-supervised learning from evolving streams.

We expect the audience of this thesis to be comfortable in the field of computer science and to have briefly read about or been introduced to machine learning.

\section{Chapter \ref{chapter:contributions}: Contributions}
In this chapter, we introduce our methodology, entitled Learning from Evolving Streams via Self-Training Novel Windowing Ensembles (LESS-TWE).

We propose a weighted soft voting scheme that uses a hyperbolic tangent function. We choose the $\tanh$ function as it allows us to approximate a logistic function while being less computationally expensive.

Next, we propose a novel windowing technique, exclusive to ensembles. Our technique allows every classifier in the ensemble to train from each data point in the stream \textbf{exactly once} similar to tumbling windows, as opposed to sliding windows where a data point is trained on \textbf{at least once}. As such, our method presents a trade-off between accuracy and execution time. Only one classifier per batch is trained on the window, therefor each classifier trains on a specific data instance at different times, resulting in delayed training. This means that only one classifier in the ensemble is trained on the newest data before the ensemble predicts labels for the subsequent batch.
The motivation for this technique is to determine if we can spend less execution time training the classifiers and to investigate how progressively delaying training of some of the classifiers in the ensemble affects concept drift detection and classification performance.

Our next contribution relates to selective self-training. In this semi-supervised learning algorithm, a classifier assigns a label it predicts to unlabelled data for future learning, if it is highly confident it is correct. There have not yet been any attempts, to the best of our knowledge, to investigate how self-training performs if labelling all unlabelled data, regardless of the classifier's confidence in its predicted label.

Finally, our last contribution is an extension of the Fast Hoeffding Drift Detecting Method for evolving data Streams. In our extension, instead of relying on the prequential accuracy, FHDDMS now makes use of an ensemble's or its classifiers' confidences, or of a boolean indicating if the classifier's and the ensemble's votes were identical for each classifier in the ensemble. Our extension allows FHDDMS to run without any labelled data, therefore making it an unsupervised drift detector.

Finally, we will cover all of our contributions by using a small toy example to explain how data is processed from a stream.

\section{Chapter \ref{chapter:experimental_design}: Experimental Design}
In this chapter, we describe how we conduct our experiments, on a MacBook Pro model \textit{11,4} running Python 3.7.3.

The data sets used for our analysis in the upcoming chapter comprise of generated SEA data with [0, 10, 20] percent noise as well as CIRCLES, SINE1 and MIXED data sets, which contain 10\% noise. Our data sets contain either abrupt or gradual concept drifts.

The estimation technique we use is prequential evaluation, also known as interleaved test-then-train, which consists of infinitely executing a loop where a classifier first predicts labels for new data (without its label), then adapts its model for said data, with the correct label. The prequential evaluation loop is provided by scikit-multiflow, a Python framework backed by A. Bifet.

The performance measures that we use are the execution time, measured in seconds, as well as the $\kappa$-temporal statistic to evaluate a classifier's predictive performance, also called $\kappa^+$ or $\kappa_t$. This $\kappa$ statistic compares our classifier to a no-change classifier and takes into account temporal dependence in the data.

Using the mean values for the entire stream for both of the metrics mentioned above, we use statistical tests to determine whether or not the differences observed are statistically significant and not due to simple coincidence. When comparing two classifiers across multiple data sets, we use the Wilcoxon test, and when comparing more than two classifiers, we use the Friedman test, coupled with the post-hoc Nemenyi test.

We conduct the following experiments:
\begin{itemize}
\item  examine the impact of each parameter value on the mean of each metric,
\item rank the results of each parameter combination in order to establish any trend regarding parameter values across the metrics,
\item compare the top ranking parameter combinations to the state of the art.
\end{itemize}

In the following chapter, we will present the results of our experiments, analyse these findings and discuss their significance.

\section{Chapter \ref{chapter:evaluation_discussion}: Experimental Evaluation and Discussion}
A preliminary examination of our results shows that our framework's execution time is, at most, very loosely tied to its predictive accuracy.

An investigation on the impact of each parameter value on the mean of each metric leads us to determine how to roughly maximise $\kappa_t$ or minimise the execution time.
The differences lie with the window type, the batch size, the percentage of labelled data used and, partially, the drift detector count. Our results indicate that using probability voting, one detector per ensemble to detect drifts, and resetting all classifiers when drifts occur would lead to better $\kappa_t$ values and a lower execution time. For the batch size, window type, and the drift detector count, it is entirely logical that choosing one value over another would change the execution time as they were, at least partially, implemented as time-saving measures.

The findings above are confirmed, and parameter values that tend to rank well are identified by ranking the parameter combinations for each metric. By examining the paired rankings (time and predictive performance), we propose an alternative parameter combination that achieves significant time savings over the one with the best predictive performance, while only reducing $\kappa_t$ by one to four percent.

Our framework runs roughly 160 times faster than the state of the art \textit{Leveraging Bagging} algorithm, and it is comparable in terms of the measured $\kappa_t$ metric. Training with only $90\%$ labelled data does not compromise our framework's predictive accuracy in comparison to \textit{Leveraging Bagging}. Our results also indicate that the predictive accuracy of our ensemble does not drop until we reduce labelled data to 80\%.

Practically, this means that our framework should definitely be considered when execution time is an important metric. However, for the applications that strictly require high predictive accuracy, Leveraging Bagging would still be the preferred choice.

%\section{Chapter \ref{chapter:conclusion}: Conclusion}




%----------------------------------------------------------------------------------------
%	BIBLIOGRAPHY
%----------------------------------------------------------------------------------------

\printbibliography[heading=bibintoc]

%----------------------------------------------------------------------------------------

\end{document}  
